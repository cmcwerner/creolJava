

%==============================================================

\section{Main Theorem (no false negatives)}

\ignore{Our definitions try to approximate infinite executions with 
  sets based on maps to finite executions. 
This makes the concept of unbounded more direct.}

%\textbf{Executions.}
In order to formally state properties of detection of flooding in
executions, 
%In order to state and prove properties of flooding,
we need some understanding of \emph{executions} and execution order, 
in particular
how method calls and completions are handled.
To capture independent processor speeds,
we  give a partial order semantics.
In order to define unboundedness in a natural manner,
we  approximate infinite executions with 
infinite  sets based on maps to finite executions. 
Other aspects such as state values,  evaluation,
and sequential execution steps are less central
and are not specified in detail. 

%At any point in an exection, the 

In our concurrency model,
each object is responsible for executing methods called on that object.
Each method instance corresponds to a \emph{process} at run time, which may be suspended by \emph{await} statements.
\OO{A started process continues until the end (given by a $put$),
 %or to an internal loop 
 to a suspension point, %(i.e., an await% with a false condition)
or to a get where the future value is not available.}
Suspended processes are kept in a queue together with incoming calls.
In the semantics below, the queues are implicitly given by 
looking at the (past)  execution history.
%\footnote{Could also stop after a satisfied await?}
An object can only execute one process at a time.
When a process is suspended, the object may continue
with another enabled process (if any).
\ignore{NOT NEEDED? We here assume that (enabled) suspended processes 
have priority over new incoming calls.}%
Local synchronous calls are done as usual, in a stack-based manner.
Local asynchronous calls are queued together with incoming calls.
%
Delays in execution are caused by delayed start of a method
execution, say when the processing object is busy,  
by  suspension, % (by await or suspending get), 
and by blocking \emph{get}s.
More formally we define system and object executions below.
\ignore{
   NEEDED?? For simplicity, we assume that resumption of suspending
   \emph{get}s  (await on future) are not unnecessarily delayed when the
   future is available; thus we do not distinguish blocking and
   suspended \emph{get}s   wrt.\ delays, both are assumed to continue without
   delay when the future value is present.}

%\subsection{Short Version}

\begin{definition}[System Execution and Object Execution]\label{def-ex}

A \emph{system execution} $E$  is a mapping from time to 
system execution at that time.
A \emph{system execution at time $t$}, denoted $E[t]$, is given by two partial mappings:
\begin{itemize} \item 
  a partial mapping from object identities to
  \emph{object executions} (one for each object in the system) at time $t$,
  describing the object execution states %/steps  
up to time $t$.
  $E[t][oid]$ denotes the execution of object $oid$ at time $t$.
%together with a set of (shared) futures, given by 
\item
  a partial mapping of future identities to values, reflecting all
  generated future values up to time $t$.
   $E[t][\mathit{fid}]$ denotes the %future 
value of $\mathit{fid}$ at time $t$. 
\end{itemize}
%\begin{definition}[Object Execution]\label{def-ex}
%
The \emph{execution of a given object %of given class 
at a given point in time} may be seen as a finite sequence of 
execution states, where
%, infinite if the execution does not terminate. 
% Each state is indexed by the corresponding node (label)
% in the program (graph), and
each state includes relevant information such as 
%future identities and 
object state and local state, as well as the program counter value
(and label in the program graph), and the unique future value of the
invocation of the each % process 
method execution.
Generated object and future identities are globally unique.
The execution %of a method 
must follow the code of the method of the class of the object,
letting the execution of a \emph{put} or an \emph{await}
take the object to an \emph{idle} state.
After an idle state the object may start a new method
execution or continue an old %suspended
 execution, after the last \emph{await},
%a release point
if the \emph{await} condition is enabled.
%
%Each object must progress with time. Thus for a system execution $E$ we have
%Thus the continuation of an idle state is is general non-determinitic.
%Each execution step must be legal, in goes ....
\end{definition}
Thus the only  non-deterministic steps appear after idle states
and at branches that depend (directly or indirectly) on the value resulting from a  \emph{get}, which depends on the environment.
%We may assume some general properties about executions%We may assume that execution continues with time
%until no further execution step is possible by any object.
We may  assume that  executions  obey the following  general properties.
\begin{description}%===================================
\item[Object Progress:]
A method execution must continue (with time)
as long as it has enabled steps.
% If the execution of  an object has an enabled step at time $t$,
%   the object must eventually continue to execute.
\ignore{
If the execution of an object at a given time ends in a non-idle state,
the process will eventually continue or end in a \emph{get} that is 
blocked at all later times.}

\ignore{is not idle
  If the execution of  an object has an enabled step at time $t$,
   each object continues to execute if possible
  %(\emph{object progress}):
  An object execution must eventually continue as long as there are 
  enabled steps (i.e., the object is idle and there are enabled processes,
  or the object is not blocked in a \emph{get} in a method execution).%
  \footnote{More formally: If an object $o$ stops at time $t$, i.e.,
  $\forall t' \,.\ t'<t \Rightarrow E[t'][o]= E[t][o]$, it must be
  blocked in a \emph{get}, and the future value is never generated, 
  or it is in an idle state with no enabled process.}
  Thus  each enabled step will be done by an object unless the object 
  gets an unbounded number of processes, in which case
  it will not be able to perform all of them.
}
\item[Historic Monotonicity:]
Executions must be \emph{historically monotonic}
in the sense that a system execution $E$
seen at two different times,  $t1$
and  $t2$ such that $t1<t2$,
must agree on the past, i.e.,  $E$ must  satisfy
% $$E[t1]\leq E[t2]$$ where $s\leq s'$ means that 
$E[t1][\mathit{fid}] \not=\bot  \Rightarrow E[t1][\mathit{fid}] =E[t2][\mathit{fid}]$
 for each future identity $\mathit{fid}$,
and
$E[t1][oid] \not=\bot  \Rightarrow E[t1][oid] \leq E[t2][oid]$ 
for each object identity $oid$ where $\leq$ denotes 
the sequence prefix ordering.
Furthermore, %Moreover,
 $E[t][oid]$ is defined if and only if
the object $oid$ has been  created in the  execution  $E[t][oid']$ of
some object $oid'$ by a  \textbf{new} statement. % Do we need to specify that the new happened ``before''?
Similarly,  $E[t][\mathit{fid}]$ is defined
if and only if
the future  $\mathit{fid}$ has been resolved in  $E[t][oid']$ by some object  $oid'$.

\item[Partial Ordering:]%
%[Partially Ordered by the ``after''-Relation]
Each system execution  must satisfy a partial ordering 
(called \textbf{after}) relating the different object executions  as
follows: 
%Each object execution %of an object $o$ (of class $C$) 
%must satisfy
\begin{itemize}
\item The first state of an object of class $C$
(other than the initial object)
 %must %be the beginning of its initiali
must occur \emph{after}
%NOTE during if synchronized ....
the object has been generated by a new $C$ construct
and must be the start of its initialization part (with an object state
reflecting class parameters). 
The object  mapping is extended
by associating the new object identity with this object execution.
\item The start of a method execution by an object $o$ must appear \emph{after} 
      the corresponding call.
\ignore{\item the next step of an execution of $o$ must  be the next statement
      in the branch of the active process (evaluating if tests and following the
      appropriate branch), unless it is an await or get statement, or local call.}
\item A blocking \emph{get} must be executed \emph{after} 
      the future value has been generated.
\item 
%a state representing the next statement 
%the state   representing the state \emph{after}
The execution of  an \emph{await} of a process $p$
must occur \emph{after} the await is enabled 
%(and when it is the next  state of the process $p$) 
and may only appear after an idle state.\footnote{Do we need to worry here about awaits that are disabled again before they run?}
\ignore{
    if the next statement is an await the remaining process (including
   the await and and the local state) is placed on the process queue of the
   object, allowing the object to proceed with another process on the
   queue, provided it is enabled.}
\item When a \emph{put} is executed (finishing the current process) 
% the current process is said to be   inished and
the future  mapping is extended by associating the produced
value with the  future identity of the current process (and we say that the future value has been
  generated). The object is in \emph{idle} state.
%\item The execution  of objects not generated (apart from the initial object)  %is $\bot$, and the future value of futures not (yet) produced is $\bot$.
\end{itemize}
\end{description}
The last two properties  may be proved by induction from a detailed operational semantics
such as that from Din et al.\cite{Din12jlap}.

\begin{definition}[Unbounded Number of Calls]\label{def-unbounded}
  A system execution is said to produce \emph{an unbounded number of
    calls} (of a certain kind) if for every bound there is a time $t$
  such that the system execution at time $t$ has a higher number of
  calls (of the given kind) than the bound.
\end{definition}

\begin{definition}[Flooding]\label{def-flooding-ex}
  An execution is flooding if it produces an unbounded number of
  uncompleted method calls.  It is flooding wrt.\ a method $m$ if it
  produces an unbounded number of uncompleted method calls to $m$.
\end{definition}


The theorem below expresses that our algorithm covers all possible floodings
(no false negatives).

\begin{theorem}[No false negatives]
\label{theorem-main}
If there is an execution of a given program which is flooding wrt.\ a
method $m$, our algorithm detects flooding of a call to $m$ or of a
call $m'$ such that flooding of $m'$ implies flooding of $m$.
\end{theorem}
\section{Proofs}

%\ignore{ move??
\textbf{Notation} of nodes: we write $call_n, put_n$, $start_n$
   where $n$ is the label of the corresponding call, and
   $get_s$  where  $s$ is a single label or a label set.
   %Similarly start nodes are indexed by the label of the corresponding call.}


In order to prove  theorem \ref{thm-flooding}
we first define a notion of execution tree, and then prove that flooding
wrt.\ a given cycle and method call 
%implies that there are uncompleted calls in the execution trees.
%We prove below that flooding according to definition   \ref{flooding-cycle} 
implies that there is an uncompleted call in some execution tree (by
lemma \ref{lemma-wr}% and \ref{{lemma-call-chain}}
), and secondly that an uncompleted call
in an execution tree implies that there is an uncompleted call in the
static graph of cycle $C$ used in the detection algorithm (by
\Blue{lemma \ref{lemma-sr}, which states that a node is SR it is in
  all execution trees of cycle $C$.}).
%\Blue{The rest of the proof follows by lemmas ....}

% We first reduce and simplify the set of
%executions without removing flooding of $m$. 
%such that for each flooding execution there is a flooding
%execution in the reduced execution set.
%
\ignore{ Consider an execution flooding wrt.\ a given call of $m$.  We
  will show that there also is an execution that does not stop before
  an enabled step (apart from start nodes) such that it is flooding
  wrt.\ the given call. We consider two cases of enabled steps,
  deterministic (unconditional) statements, and (blocking or
  non-blocking) statements depending on a future (\emph{get}).

   We first observe that for an execution not including such a step,
   the execution extended with the step at some time $t$ when the step
   is enabled, is a possible execution according to the partial order
   semantics (which describes the possible executions).

%1. Deterministic steps
   \begin{description}\item[Deterministic statements]
   (assignments, simple calls, local calls, object generation, and skip):
   \footnote{The partial order semantics describes the possible executions.  It
   captures independent object speeds.  It is obvious that for
   non-interfering, deterministic object execution steps, 
%except when waiting is involved, 
   the final outcomes are the same, regardless of how fast these are executed.
%One may therefore move  non-interfering, deterministic object
%execution steps forward (as long as they are enabled).

    We may pretend that deterministic execution steps take no time
   (not changing the time of  execution of start nodes and continuations 
    of \emph{get}s and \emph{await}s).
}
%of a blocking or suspending wait.
This \emph{simplification} will mean that some calls are produced earlier 
%(which is OK with respect to flooding, i.e., 
%flooding cannot disapear by earlier occurrence of calls), 
and that some \emph{put}s will appear earlier.
This will not cause flooding of $m$ to disappear,
because flooding cannot disappear by earlier
occurrence of calls, and not by earlier \emph{put}s
%The latter will not cause flooding of $m$ to disappear
 because there are
only finitely many objects $y$ that can execute $m$.
Therefore, to find a time where the simplified
execution has  more than $x$ uncompleted $m$ calls,
% in the modified execution, 
we take an original execution at a time where there are $x+y$
uncompleted $m$ calls, % in the origial execution, 
which exists by the assumption of unbounded number of uncompleted $m$ calls.
%
Thus we may assume that at any time $t$ an object $o$ is either idle
or is waiting in a blocking \emph{get}.

%2. gets:  (get or await)
\item[\emph{get} statements in a method other than $m$]
%Consider first method executions other than $m$.
Blocking \emph{get}:
In this case, the  execution of a specific  get is never done,
even after being enabled.
%Doing this step  will not affect $m$ executions if 
By doing the step in a method other than $m$,
$m$ executions  are not affected.
%\item[Suspended \emph{get} statements] 

   Suspended get in a method other than $m$: If done by an object
   being flooded with $m$, $m$ executions may be delayed, for other
   objects $m$ executions are not affected.  In either case, flooding
   is not reduced.  If the object is deadlock...
   \end{description}

   \footnote{(Note that wrt.\ await on condition:
   the algorithm will not know that such conditions are enables,
   and will treat them as unabled.)}
================================= ignore =============================}%
%
\ignore{Thus it suffices to consider executions such that
the \emph{after}-relation is giving priority to active processes as long as
the next step is enabled (i.e., it is not in a blocking get and
unresolved get/await).  Thus a new call is started after all active
processes have reached an idle state or a blocked get (or end of loop,
if the cycle is a loop).
%This is a benefit of the partial order semantics!
Similarly
we may also assume that resumption of suspending \emph{get}s  (await on a
future) are not unnecessarily delayed when the future is available;
thus we do not distinguish blocking and suspended \emph{get}s  wrt.\ delays,
both are assumed to continue without delay when the future value is
present. }
%==========end of reduction of possible executions
%
%\textbf{Note:} In the discussion below we restrict ourselves to simplified executions.
%



Consider a system execution $E$ %, flooding wrt.\ $m$ and
with unbounded iterations of a cycle $C$.
We define the execution tree of iteration i, denoted $T_{E,C,i}$, reflecting 
all executions in $E$  caused by %$C$  during 
 iteration $i$ of the cycle, 
considering all % one or more 
involved objects.

\begin{definition}[Execution Tree]\label{def-tree} 
The execution tree $T_{E,C,i}$  of a system execution $E$ with unbounded  
iterations of the cycle $C$ 
%and with unbounded number of calls to $m$,
is  obtained from  $E$ 
% given by 
by % taking the $i$th iteration of the cycle,
taking the start node of the  $i$th  iteration as the root,
and  including 
\begin{enumerate}
\item \label{def-tree-cycle}
all steps in the cycle iteration except the back edge
\item \label{def-tree-method} \OO{all steps  in $E$
ordered \emph{after} a step already in the tree, according to the 
partial \emph{after} ordering (letting call edges form the branches)}
\item  \label{def-tree-flows} \OO{
the continuation of processes with states already in $T$ over time.}

\ignore{ all states that happen \emph{after} (according to the assumed
  semantic partial order) a node all ready in T and before some other
  state already in T,}
\ignore{ all method executions in $E$ directly or indirectly called
  from the iteration at some time. (These executions form the
  branches).}
\ignore{
  \item \label{def-tree-flows} all steps in these method executions
  as long as  each  \emph{get} or  \emph{await}  is resolved within 
  the tree or by completions before the start of the iteration.}
  % restricted to calls caused by that iteration%
\end{enumerate}\end{definition}
   \ignore{ The execution tree $T_{E,C,i}$ is the set of trees extending 
    the cycle $C$ (minus the back edge) %(without the cycling call) 
    with possible paths through off-cycle parts 
    and with some  invocations, such that
   \begin{enumerate}
   \item each method execution ends at a $put$, or just before an  $await$ or
    a  $get_l$ when the corresponding $put_l$ is not in the tree,
    %at a $get_k$ when $call_k$ is not completed (see below) or at an await,
    and finally
    \item for each $get_l$ in the tree, there is a  $put_l$ in the tree.
   (Unlabeled \emph{get}s  are possible ...)

  \ignore{all states in a process that already has states in T up to the
   but not including a blocked get or await, or through the end of the
   process/method including the put, whichever occurs first.}
   \end{enumerate}}
Clearly  for every start node in
the tree (apart from the root) there is a matching call in the tree.
Flooding calls will in general not be in all  executions  $E$
and therefore not in all execution trees.
%
The tree may be infinite (in case of inner, direct or indirect,
recursion in the cycle) since we do not limit the time. However, we
will be interested in properties of textual occurrences of calls,
and there will only be a finite number of these. %labels involved.
%
Note that according to the object progress assumption, 
  \ignore{all the active method execution of an object ends in an
    unresolved \emph{get} or an \emph{await}.  Enabled processes are
    performed unless there is an unbounded number of them. Thus
    execution trees have a third property
% \begin{enumerate}\setcounter{enumi}{2}
 \item \label{def-tree-flows}}%
\OO{all method executions partly in the cycle must terminate or
 end in an unresolved  \emph{get} or  an \emph{await}.
 It does not ensure that called methods will eventually start to execute.}
\ignore{
   Since there is a finite number of objects involved in the cycle,
   these cannot be blocked. The next cycles would then be blocked.}%
%\end{enumerate}
%
\ignore{
each method execution will end with an unresolved \emph{get} or
\emph{await},
%unless the object has an unbounded number of processes.
\begin{itemize}
\item \label{def-tree-flows} 
all started method executions end in an unresolved
 \emph{get} or  \emph{await}  
     \ignore{ all steps in these method executions as long as each
     \emph{get} or \emph{await} is resolved within the tree or by
       completions before the start of the iteration.}
\end{itemize}}
%objects involved in the cycle....

%Consider all possible system executions. 

%\subsection{ADJUSTMENTS MAY BE NEEDED FROM HERE}
\ignore{%==============================================================
\subsection{Long Version}


\begin{definition}[Object Execution Trace]\label{def-object-ex}
An \emph{execution trace of a given object} of given class at 
a given point in time may be seen as a  finite 
%or infinite 
sequence of execution states, where
%, infinite if the execution does not terminate. 
% Each state is indexed by the corresponding node (label)
% in the program (graph), and
each state includes relevant information such as 
%future identities and 
object state and local state, as well as the program counter value (and label in the program graph), 
and the unique future value of the execution of the current process.
%We say that a state is an s-state if labeled by a statement s.
\ignore{It has a process queue, consisting of (internal and external)
  method executions and remaining parts of processes starting with an
  await on a condition or a future.  A process on the queue is enabled
  if the it starts with an await on a condition that evaluates to
  true, or with an await on a future.  A condition is enabled ...}
The execution reflects method executions performed by the object, and 
must satisfy some basic properties concerning first and next states:
\begin{itemize}
\item
It must start with a state labeled with
the first statement of %a branch through
 the initialization part of the class
(which is the only time this initialization appears in the execution of that object).
\item
The next state after a state with a given program counter value
%The next step after an execution of a statement 
must have the counter value of the next statement
(evaluating if-tests and following the
appropriate branch), 
unless it is a put statement or an await on a condition or a future,
 in which case the next state is the \emph{idle} state,
or a local call, in which case the next state 
has a counter value which 
corresponds to the first statement of the called method;
and in this case the next statement after the end of the 
execution of the locally called method, is the statement following
the local call.

%\item \Blue{After return from a local call, the next statement is the 
%next after the call statement ....}
\item
The next state after idle may only be a start of a  method (of C)
or a state representing the next statement
after a previous await state
of an uncompleted process.
%corresponding to methods performed by the object.
\end{itemize}\end{definition}
Execution of (blocking) \emph{get}s  and idle states 
depends on the  execution of  the whole  system as explained below.

%========================= System Execution  =================================
\begin{definition}[System Execution, System Trace, and Monotonicity]\label{def-system-ex}
An \emph{execution of an object system} (including one initial  object)  is defined by 
\Blue{a set of historically monotone system traces, each reflecting the execution %of each object 
up to  a given time.
As defined below the monotonicity requirement  expresses  that they agree on past traces.
A \emph{system trace} is a partial mapping from object identities to}
object execution traces, one for each object in the system,
together with a set of (shared) futures, given by a partial mapping of
future identities to values (reflecting generated future values), and
with a partial ordering 
(\emph{after}) relating the different object execution traces as
follows: The execution trace of each object $o$ must satisfy
\begin{itemize}
\item the first state of an object of class $C$
(other than the initial object)
 %must %be the beginning of its initiali
must occur \emph{after}
%NOTE during if synchronized ....
the object has been generated by a new $C$ construct,
and must be the start of its initialization part (with an object state
reflecting class parameters)
\item the start of a method execution must
appear \emph{after} the corresponding call, and the call must be on $o$
\ignore{\item the next step of an execution of $o$ must  be the next statement
in the branch of the active process (evaluating if tests and following the
appropriate branch), unless it is an await or get statement, or local call.}
\item a blocking get must be executed \emph{after} the future value has been generated
\item 
a state representing the next statement \emph{after} an await of a process $p$
must occur \emph{after} an idle state when the await is enabled (and when it is the next 
state of the process $p$). % and when the previous state reaches an idle state.
\ignore{
if the next statement is an await the remaining process (including
  the await and and the local state) is placed on the process queue of the
  object, allowing the object to proceed with another process on the
  queue, provided it is enabled.}
\item when a put is executed (the current process is said to be
  finished and) the future  mapping is extended
by associating the produced
value with the   future identity (and we say that the future value has been
  generated). The object is in \emph{idle} state.
\item The execution trace of objects not generated (apart from the initial object)  is $\bot$,
and the future value of futures not (yet) produced is $\bot$.
\end{itemize}
The set of system states must be historically monotonic
in the sense that any two  system states $s1$ and $s2$ must satisfy $s1\leq s2$ or  $s2\leq s1$,
where $s1\leq s2$ means that 
$s1[oid] \not=\bot  \Rightarrow s1[oid] \leq s2[oid]$, for each object identity $oid$
and $s1[\mathit{fid}] \not=\bot  \Rightarrow s1[\mathit{fid}] =s2[\mathit{fid}]$, for each future identity $\mathit{fid}$.
\end{definition}
%
\ignore{OLD: 
An infinite execution is captured by an unbounded set of finite execution sets
(one for each object)
such that the set is prefix-closed (i.e.,  the set  of all finite initial parts of the execution).
We say that an execution is flooding if
An infinite execution is captured by an unbounded set of finite executions
such that the set is prefix-closed (i.e.,  the set  of all finite initial parts of the execution).}

\begin{definition}[Unbounded Number of Calls]\label{def-unbounded}
A  system execution is said to produce \emph{an unbounded number of calls} (of a certain kind)
if for every bound it has a trace with a higher number of  calls (of the given kind).
\end{definition}

\begin{definition}[Flooding]\label{def-flooding-ex}
An execution is flooding if it produces an unbounded number of
uncompleted method calls.
It is flooding wrt.\ a method $m$ if it produces an unbounded number of
uncompleted method calls to  $m$.
\end{definition}


The theorem below expresses that our algorithm covers all possible floodings
(no false negatives).

\begin{theorem}[No false negatives]
\label{theorem-main}
If there is an execution of a given program which is flooding wrt.\ a method $m$, 
our algorithm detects flooding of a  call to $m$ or of 
a call  $m'$ such that flooding of $m'$ implies flooding of $m$.
\end{theorem}

\textbf{Notation}
of nodes: we write $call_n, get_n, put_n$, and $start_n$
where $n$ is the label of the corresponding call. 
%Similarly start nodes are indexed by the label of the corresponding call.


%\textbf{Proof: }
\begin{proof}
Assume that there is an execution $E$ of a given program with flooding
wrt.\ a method $m$.  We need to prove that flooding of a call to $m$
is detected (or at least a call $m'$ such that flooding of $m'$
implies flooding of $m$).


Consider an execution $E$ which is flooding  wrt.\ to method $m$.  Since
there are finitely many $m$ calls in the code, the execution must also
be flooding  wrt.\ one of these calls, say labeled $n$.  Thus it suffices
to consider each textual call rather than each method.  Let us assume
that $call_n$ is flooding in given execution $E$.
%%For  $E$  the call $call_n$
This call must be caused by a cycle $C$, repeated without bound.  (%In $E$ 
The cycle may be interleaved with other flooding or non-flooding
cycles.)  Thus the call $call_n$ must occur in a method
which is called (directly or indirectly) from an iteration of  the cycle,
i.e., \emph{there must be a call path from the cycle to the call}.
%the call  $n$  may be off-cycle).
%
\textbf{Note:}
Thus flooding wrt.\ $m$ in the sense of \ref{def-flooding-ex}  implies 
flooding wrt.\ a cycle in the sense of \ref{flooding-cycle}\,. 


Consider a repeated cycle involving one or more processes.
At the time when the next iteration of the cycle starts there may be some remaining parts
of processes started in the previous  iteration of the  cycle. 
%Since these are processing have the any remaining bort
Flooding occurs when an object over time gets more processes to do  
than it can handle. Each iteration of a cycle may
generate new calls on some objects and may leave some unfinished
processes. However, the unfinished processes are already active,
and new processes on that object  generated in  the next cycle cannot start
until the previous ones are finished or suspended.
Assuming there is a bounded number of objects, the same objects will
be involved in  future cycles, and if one of these %that object 
is busy, the cycling will need to wait.
\textbf{Note:} Thus flooding due to blocking \emph{get}s  is not a problem for objects
involved in the cycle.\footnote{
   NOT NEEDED?  Thus flooding can be caused by called methods
   being delayed, or by suspended parts of remaining processes
   being disabled.}%
\footnote{DO WE NEED THIS? 
   Thus it suffices to look at calls generated by a cycle and remaining
   parts of processes \emph{after} suspension.  Remaining processes without
   suspension do not cause flooding.  And suspension while waiting for
   a future can not cause flooding if the future value is available
   (assuming that a suspending get has priority over new calls when
   enabled).  Thus flooding may be caused by calls generated by the
   cycle, remaining parts of processes after an await or a get on a
   future not ready.  }%
\footnote{MAY NOT NEED ANY LONGER? We may need to assume there is no
  deadlock.  With respect to possible executions: we assume that a
  method execution that has started continues until the end or to an
  internal loop or to an await with a false condition or to a get
  where the future value is not available.  A called method may be
  delayed (reflecting that process speeds are independent and may also
  depend on scheduling).}
%
\ignore{REUSE DEFINITION IN NEW CONTEXT, SKIP OTHER PARTS: 
 The cycle $C$ can be seen as a sequence of nodes starting with a start
 node of some method $p$ (or start node of a loop) following one
 possible path through that method (loop). At a call it may continue
 with the start node of the called method (following one possible path
 through that method) or continue with the path through $p$, and so on
 (involving zero or more methods calls), until it reaches a call back
 to $p$ (or end of the loop body).  There may be some remaining parts
 of the methods started during the cycle, called \emph{off-cycle part}
 (e.g.\ $put:2$ in fig.\ \ref{graph-modified}), and there may be some
 method calls leading out of the cycle (e.g.\ $Pd:1$ in
 fig.\ \ref{graph-modified}) . % ("off-cycle calls").
 The flooding call $call_n$ must be in the cycle, in off-cycle parts, or 
 in methods directly or indirectly called from the cycle or off-cycle parts. 

 Consider the set of execution trees $T_{C,i}$ reflecting possible
 executions during one iteration $i$ of the cycle, considering one or
 more involved objects.  Execution of called methods are included if
 they are started during the cycle (before the next iteration) and
 suspension is ignored if resolved during the cycle iteration.
 \begin{definition}[Execution tree]\label{def-tree}
  The set of execution trees $T_{C,i}$ is the set of trees extending
  the cycle $C$ (minus the back edge) with possible paths through
  off-cycle parts and with some  invocations such that
 \begin{enumerate}
  \item $call_n$ (the call causing the flooding) is in the tree, and 
  \item for every start node there is a matching call in the tree,    and  
 %(apart from the initial start node of $p$) %every invocation is called from the  tree,
  \item each method execution ends at a $put$, or just before an
   $await$ or a $get_l$ when the corresponding $put_l$ is not in the
   tree,
 %at a $get_k$ when $call_k$ is not completed (see below) or at an await,
  and finally
  \item for each $get_l$ in the tree, there is a  $put_l$ in the tree.
 \end{enumerate}\end{definition}
 The first point says that we are only considering flooding iterations;
 the second that only relevant calls are considered; the third that an
 unfinished execution must be in an unresolved waiting condition; and
 the last point says that the tree is self-contained in the sense that
 it includes all method executions known to complete, and all nodes of
 the chosen path of that execution must be in the tree, including
 $get$s to other methods (which may in turn mean that yet other
 \emph{put}s must be in the tree).  A call $call_l$ without a start
 node in the tree (in which case there is no $get_l$ in the tree),
 reflects that the call was delayed until after the cycle.
 = END IGNORE =====================}

%\textbf{Note:} 
The partial order semantics describes the possible
executions.  It captures independent object speed.  If an execution is
flooding with respect to a method $m$ for some execution $E$, it is
also flooding in executions where the starts of executions of $m$ are
 (more or less) delayed.  
\Blue{It is obvious that for non-interfering, deterministic object
execution steps, the final outcomes are the same, regardless of how
these are ordered.}  Thus it suffices to consider executions such that
the \emph{after}-relation is giving priority to active processes as long as
the next step is enabled (i.e., it is not in a blocking get and
unresolved get/await).  Thus a new call is started after all active
processes have reached an idle state or a blocked get (or end of loop,
if the cycle is a loop).
%\footnote{this is a benefit of the partial order semantics!}
Similarly
we may also assume that resumption of suspending \emph{get}s  (await on a
future) are not unnecessarily delayed when the future is available;
thus we do not distinguish blocking and suspended \emph{get}s  wrt.\ delays,
both are assumed to continue without delay when the future value is
present. \textbf{Note:} In the discussion below we restrict ourselves to such 
executions.



Consider a system execution  $E$ with unbounded iterations of a cycle $C$.
We define the execution tree of iteration i, denoted $T_{E,C,i}$, reflecting 
all executions in E caused by C during  iteration $i$ of the cycle, 
considering one or more involved objects.
\footnote{OLD TEXT  - REDO:
 Execution of called methods (identified by the future identities  of the calls) are included if they are started 
 in $E$
% during   the cycle (before the next iteration) 
and suspension is ignored if resolved during the cycle iteration.}
 
\begin{definition}[Execution Tree]\label{def-tree} 
(MAY NEED TO ADJUST A BIT MORE!)
The execution tree $T_{E,C,i}$ 
 of a given system trace $E$
 with at least $i$ iterations of the cycle and with at least
 $i$  calls to the  flooding method,
 is the system trace $E$ 
 given by the $i$th iteration of the cycle
 (without the start of the next iteration), 
 restricted to calls caused by that iteration%
 \ignore{  The execution tree $T_{E,C,i}$ is the set of trees extending 
  the cycle $C$ (minus the back edge) %(without the cycling call) 
  with possible paths through off-cycle parts 
  and with some  invocations},
such that
\begin{enumerate}
% \item $call_n$ (the call causing the flooding) is in the tree, and 
 \item for every start node there is a matching call in the tree,  %(apart from   the initial start node of $p$)
  %every invocation is called from the tree,
  and  
 \item each method execution ends at a $put$, or just before an  $await$ or
  a  $get_l$ when the corresponding $put_l$ is not in the tree,
  %at a $get_k$ when $call_k$ is not completed (see below) or at an await,
  and finally
  \item for each $get_l$ in the tree, there is a  $put_l$ in the tree.
(Unlabeled \emph{get}s  are possible ...)
\end{enumerate}

\end{definition}
Note that 
$call_n$ (the call causing the flooding) is in the tree,
(since %The first point says that 
we are only considering  flooding iterations).
the first point says  that only relevant calls are considered;
the second that an unfinished execution must
be in an unresolved waiting condition; and 
the last point says that 
the tree is self-contained in the sense that it includes
%we must include 
all method executions known to complete,
and all nodes of the chosen path of that execution must be in the tree,
including $get$s to other methods (which may in turn mean that yet other
\emph{put}s must be in the tree). 
A call $call_l$  without a start node in the tree
 (in which case there is  no $get_l$ in the tree), reflects that 
the call was delayed until after the cycle.
\footnote{since in the set of traces we may stop before a start node.}



A tree will be finite since there are finitely many labeled calls,
and each labeled call can only give rise to one 
start node and corresponding  path through the method.

}%=====================end ignore LONG VERSION ==========================

\begin{definition}[Execution Tree Set]\label{def-tree-set}
 The set of execution trees
 $T_{C}$ 
is the set of execution trees $T_{E,C,i}$ 
for all possible system executions $E$ 
with an unbounded number of $C$ iterations, and for any possible iteration $i$.
%traces E of a system execution  with an unbounded number of C iterations,
%and for any possible iteration i.
\end{definition}
%
\ignore{MERGE WITH EARLIER TEXT??
The set of possible trees reflect
that continued execution of the remaining part of a  method 
is fast as long as there is no suspension (i.e. no time).
 Suspension may take no time or non-trivial time. Similarly
execution of a new method may take no time or non-trivial time.
Thus an execution tree may or may not be cut at any such  point.
We assume here that a get on a completed call takes "no time".}
%may or may not take Since time needed in cases of suspension or
%invocation depends on the execution and may range from zero time to
%non-trivial time.
\ignore{ The set of trees $T_{C}$ represents possible executions of an
  iteration of cycle $C$ where method executions not in the tree are
  delayed.  For each iteration of of the cycle in an execution, there
  will be a tree, reflecting the activity related to the iteration up
  to the point where the next iteration starts.
     A call with label $k$ is said to be \emph{completed} in a tree
  $T_{E,C,i}$ if a $put_k$ or $get_k$ is in the tree.
}% end ignore
%
A call with label $k$ is said to be  (possibly) reachable if  
$call_k \in %%%%%%%%%%%%%%%%\bigcup_{E,i}  
T_{E,C,i}$ for some $E$ and $i$.
A call $k$ is (surely) completed if 
$put_k \in %%%%%%%%%%%%%%%\bigcap_{E,i}  
T_{E,C,i} $ for all  $E$ and $i$.\footnote{Or $get_k$?
No, there must be a put before get}
% or  $get_k\in \bigcap_i T_{C,i} $
We define $\callsE$ by the label set $\{n \,|\, \exists\ E,i \ .\  call_n\in T_{E,C,i}\}$, and 
$\compsE$ by $\{n \,|\,  \forall\, E,i\ .\  put_n\in T_{E,C,i}\}$.

We next show that  $ \callsE$ is 
%the same as the 
 %overapproximates 
contained in the set of  calls   detected,  
and  that all detected  $comps$ are in 
$\compsE$. % is the same as  the detected $comps$.
%(TODO: this is  not quite true since we may detect a call m' such that
%flooding of m' implies flooding of $call_n$).
Thus if 
a call $n$  is flooding in some execution,
$n$ is in  $ \callsE-\compsE$, and thus
in  $ \callsD-\compsD$, i.e., the flooding call  must be detected.




\ignore{====================OLD=============
    In the analysis below we look at reachable nodes in the graph
    $G_{C,E}$ consisting of the cycle $C$ (following only the chosen
    path where there is branching), remaining method parts (including
    all paths), and also called methods.  The nodes of $G_{C,E}$ are
    reachable from the cycle and the set of calls in $G_{C,E}$ is
    called the reachable calls of the cycle, denoted $ calls_{C}$.

   $call_n\in G_{C,E}$

   $\callsE= \{n| call_n \in G_{C,E}\}$

   For given execution E and cycle C, we may look at all calls
   reachable from an iteration of the cycle, say $calls_{E,C}$. We
   must have $n\in calls_{E,C}$ And if we look at all completions,
   $comps_{E,C}$, that are made in every iteration of the cycle, we
   must have $n\not\in comps_{E,C}$.  I.e., the call is in the cycle
   or reachable from the cycle, and the call is not completed an
   unbounded number of times.


   We next define $ calls_{C}$ and $comps_{C}$ such that the former
   over-approximates $ calls_{E,C}$ and the latter under-approximates
   $comps_{E,C}$.  Thus if $n$ is in $ calls_{E,C}-comps_{E,C}$ it
   must also be in $ calls_{C}-comps_{C}$.

   $ calls_{C}$ consists of all calls $n$ such that

  \begin{itemize}
  \item call $n$ is in $C$
  \item  call  $n$  is in some  off-cycle path of $C$
  \item call $n$ is in some method called by $C$ (directly or
  indirectly)
  \end{itemize}
   $ comps_{C}$ consists of all calls $n$ such that
  \begin{itemize}
  \item put  $n$  is in $C$
  \item get  $n$  is in $C$
  \item put/get $n$ is reachable in each path of an off-cycle path of $C$
  \item put/get $n$ is  in each path of a method in $ calls_{C}\cap  comps_{C}$
  \end{itemize} %if a call i is in the cycle it is in
}
\ignore{Note:
 consider the cyclic program
 m(){if ... then f:= call m1 else f:= call m2 fi;
    get f; call m (): put}
 Here we would need two put-get edges, one from m1 and one from m2
}
\footnote{\textbf{Note:} There could be a node $get\, f$ in the cycle
  where the future f is defined outside the cycle, say by a parameter
  or a field.  Will this be a Problem for our definitions?  \Blue{Such
    a $get\ f$ can be handled as an $await$ since completion cannot be
    detected.}  }
%{\textbf{Note:}}
\ignore{
\subsection{MAY NOT NEED:  Alternative  Executions}
\emph{Motivation:  Rather than trees $ T_C,j$
  we  consider initial parts of executions
  (which is the real thing when it comes to semantics).}

   An \emph{execution} can then be seen as a finite (if the execution
   stops) or infinite (if the execution does not stop) sequence of
   states, augmented with relevant information about executing nodes
   etc.  Flooding of a cycle $C$ wrt.\ a method $m$ would mean that
   there are one or more infinite executions where $C$ is repeated
   infinitely and there are infinite number of calls to $m$.

   We assume that each process has no infinite sequence of execution
   steps until the process reaches a \emph{resting} state,
   i.e.\ process end, a blocking get, or a suspension point (await).

   %If the call back to the cycle is a local, blocking call 
   \ignore{ For such an execution $E_C$ we let $E_C,n$ mean the
     finite, initial part of $E_C$ ending just before the $n+1$th
     iteration of $C$, letting each object (apart from the one doing
     the start of the cycle -- in case of recursion?) reach a resting
     state.  (If there is flooding of $C$ wrt.\ a method $m$, there is
     also flooding of $C$ wrt.\ a method $m$ for an execution of this
     kind, since all possible object speeds must be considered.)}  }

\ignore{We may continue more or less as before.  For each
  iteration/occurrence of the cycle $C$ in such an execution $E$, we
  define the associated execution tree $T_{E,C}$ which contains the
  occurrence of the cycle $C$ extended with call edges and start nodes
  for called methods starting to execute before the next iteration of
  the cycle, letting each execution end in a waiting, blocking or
  ending node.  By induction on the set of execution we know that ...
  same def as before (more or less)...  to be completed.}


%\paragraph{Observation:}
\begin{lemma}
\label{lemma-await}
 We may ignore executions where await on a future is 
treated different than get, in the sense that the set of floodings
in the first case is not larger than the set of floodings in the 
latter case.
\end{lemma} 
\begin{proof}
We first observe that 
with the under-specified scheduling of our partial order semantics,
a possible way of implementing suspending gets is
by implementing them as a (blocking) get.
It then suffices to prove that 
executions different from those using a blocking get,
do not provide more flooding.

A method execution of blocking rather than suspending
may contribute to flooding since it may not terminate.
However, if there are unbounded many calls to such methods,
the method executions can be delayed before start, rather than
at an await. 
In case the there is a call back in the first part of the method
(before a suspending get), 
this may or may not be the call back of the current cycle $C$.
In the first case, the executing object 
the completion of the current call is not strongly reachable,
so flooding in this case is detected.
In the second case,
we have a shorter cycle that is flooding.
Therefore  flooding of this method is already possible.

A call $n$ occurring after a suspending get, may never be
performed with executions that perform  suspending gets by blocking.
% rather than suspend. 
 However, these calls will be weakly reachable
and thus count in the static  detection of flooding.
The same argumentation applies to calls indirectly caused by %the 
call $n$.
%path starting after a suspended get.


Anything else...

\ignore{
Consider an execution $E$ where the execution of 
an await on a future in a given process %method instance 
is reached at time $t$.
It could be that the await is never enabled,
or that it is  enabled at time $t'>t$
(in which case the process will continue to be enabled until executed).
In the latter case 
the await may 
never be executed or it may be executed at time  $t''>t'$.
Compare this execution to an execution $E'$ where the await statement
is executed like a blocking get.
%, at some time $t`>t$, and such that the two exeutions are
%equivalent up to time $t'$.

If the process represent the flooding method ...}
\end{proof}



\begin{lemma}
\label{lemma-sr}
If node $N_i$ is \emph{strongly-reachable} (SR) in graph $G$ with respect to cycle $C$ then $N_i$ is in every $T_{E,C,j}$ as defined in 
definition \ref{def-tree}\,.
\end{lemma}

\begin{proof}
We will use structural induction on the set of SR nodes to prove that Lemma \ref{lemma-sr} 
holds for all SR nodes as defined by definition \ref{defn-sreachable}\,.
\\ \emph%\subsubsection
{Base case:}
Let the set $S$ of SR nodes contain just nodes that are in $C$ (defn: \ref{defn-sreachable}.\ref{sr-incycle}).
Clearly all nodes in $S$ will be in $T_{E,C,j}$ by definition of $T_{E,C,m}$ (definition \ref{def-tree}\,).\footnote{What is $m$ in  $T_{E,C,m}$}
%
\\ \emph%subsubsection
{Induction step:}
Assume Lemma \ref{lemma-sr}  holds for some set $S$ of SR nodes. We will show that applying each of the subparts of defn: \ref{defn-sreachable}
defines a node to be SR for which the lemma holds.

\ref{defn-sreachable}.2: $N_j$ is not a blocking node. $N_i$ is in every $T_{E,C,k}$ by the inductive assumption. 
The only way to continue execution of the method containing $N_i$ is to execute $N_j$. Therefore 
$N_j$ must also be in $T_{E,C,k}$ (definition \ref{def-tree}.c).

\ref{defn-sreachable}.3: The argument is the same as for the previous case with the addition of the assumption that the \emph{put}
indicating the completion of any possible future needed by this \emph{get}
must also be SR (and in the tree). Since the only way to continue the execution of the method containing $N_i$
is $N_j$ and $N_j$ is unblocked due to the completion of its future, $N_j$ must be in $T_{E,C,i}$ (definition \ref{def-tree}.c).

\ref{defn-sreachable}.4: From the assumption of $n \in {\compsD}$, the method corresponding to
${start}_n$ has finished and it cannot finish without starting, thus
the node ${start}_n$ must have executed before the end of the tree. 
Also by assumption, the corresponding node ${call}_n$ is SR and in $T_{E,C,m}$, thus the node ${start}_n$ must also
be in $T_{E,C,m}$, satisfying the lemma.

\ref{defn-sreachable}.5: The nodes ${start}_n$ and $N_k$ (a ${get}_n$ node) are SR and in $T_{E,C,m}$ (by assumption). But ${get}_n$ can only execute
if ${put}_n$ has executed, thus the ${put}_n$ node $N_j$ must also be in $T_{E,C,m}$.

\ref{defn-sreachable}.6: The only way to get to node $N_k$ is through $N_j$ and by assumption $N_k$ is SR and in every $T_{E,C,m}$.
Thus the only way for $N_j$ to not be in $T_{E,C,m}$ would be for $N_k$ to be the first node in its method in $T_{E,C,m}$.
But this is impossible.
$N_k$ cannot be a \emph{start} node as \emph{start} nodes are not flow-reachable from any node. 
If $N_k$ is the first node in its method to execute in $T_{E,C,m}$ (i.e. although we must go through $N_j$ to get to $N_k$ maybe
$N_j$ is not in $T_{E,C,m}$), $N_k$ must be in the cycle which implies some other node $N_i$ reaches $N_k$ (to close the cycle) 
which contradicts the assumption that $N_k$ is not flow-reachable from any node other than $N_j$.
\end{proof}

\begin{lemma}
\label{lemma-call-chain}
If node $call_n$ is \emph{flooding} with respect to cycle $C$ in graph $G$ then there must be a call chain from a \emph{call}
in $C$ to $call_n$.\footnote{
Moreover, the start node of the  method called by e $call_n$
 is weakly reachable.}
\end{lemma}

\begin{proof}%Proof: 
This %The first part of this 
lemma follows directly from definition  \ref{flooding-cycle}\,.
\footnote{The second part follows by simple inductin on the length of the call chain
using  definition \ref{defn-wreachable}.\ref{wr-calledge}\,.}
\end{proof}

\begin{lemma}
\label{lemma-no-flood-chain}
If $\exists$ a call chain ${call}_0, {call}_1, ... {call}_n$ such that ${call}_0 \in C$ and all ${call}_i, i <= n$ are not flooding,
then all ${start}_i, i <= n$ are SR.
\end{lemma}

\begin{proof}%Proof: 
We will use structural induction on the length of the call chain.\\
\emph{Base case:} If the length of the call chain is zero, then $call_0$ is SR by definition \ref{defn-sreachable}.\ref{sr-incycle}\,.
Because $call_0$ is not flooding, $n$ must be in $\compsD$. Therefore
${start}_0$ is SR by definition \ref{defn-sreachable}.\ref{sr-start}\,.
\\
\emph{
Induction step:} If the length of the call chain, $n$, is greater than zero, then assume all ${start}_i, i < n$ are SR.
However, the only way for ${start}_i$ to be SR is for ${call}_i$ to be SR (definition \ref{defn-sreachable}.\ref{sr-start}), therefore
all ${call}_i, i < n$ are SR. Because ${start}_{n-1}$ is SR then by definition \ref{defn-wreachable}.\ref{wr-start} ${call}_n$ is weakly
reachable and thus $n \in {\callsD}$, and from the assumption of the lemma, ${call}_n$ is not flooding,
it must be the case that $n \in {\compsD}$. 
Therefore, by definition \ref{defn-sreachable}.\ref{sr-start}, 
${start}_n$ is SR.
\end{proof}

\begin{lemma}
\label{lemma-wr}
If node $call_n$ is \emph{flooding} with respect to cycle $C$ in graph $G$ then $n \in {\callsD}$\ignore{THE FOLLOWING PART CAN NOW BE REMOVED, RIGHT? or
$\exists call_i$ that is flooding, there is a call chain from $call_i$ to $call_n$, and $i \in {\callsD}$}.
\end{lemma}

\begin{proof}%Proof: 
We will use  induction on the length of the call chain from the call $call_0 \in C$ to $call_n$, where the
call chain is $call_0, call_1, ... call_n$.
\\
\emph{Base case}: 
If the length of the call chain is zero, then $call_0$ is the flooding
call which by definition \ref{defn-sreachable}.\ref{sr-incycle} is SR
and thus by definitions \ref{defn-wreachable} and \ref{defn-calls}, $0
\in {\callsD}$.
\\
\emph{Induction step}: 
If the length of the call chain is $n+1$ (with $n\geq 0$),
then we may assume $n \in {\callsD}$.
By definitions  {\ref{defn-wreachable}.\ref{wr-calledge}}
and {\ref{defn-wreachable}.\ref{wr-start}}
we have that ${n+1} \in {\callsD}$.
\ignore{OLD TEXT WITH OLD DEF OF WR:
  If the length of the call chain, $n$,
  is greater than zero,  then assume 
  $i \in {\callsD}$ for all $call_i$ in the chain $i < n$.
  %It there is an $i<n$ such that $call_i$ is flooding the lemma holds.
  %Otherwise, 
  \Blue{We must then prove $n\in \callsD$.  There are two cases to
   consider.  First, if there is an $i<n$ such that $call_i$ is
   flooding then by the inductive hypothesis $i \in {\callsD}$ and there
   is a call chain from $call_i$ to $call_n$, satisfying the lemma.}
  In the other case, $call_n$ is the first call in the chain to be
  flooding.  By assumption the method with $start_{n-1}$ contains the
  call $call_n$.  $\forall i < n, {call_i}$, is not flooding thus
  $call_i$ is SR. \Blue{(I forget why not flooding implies SR. I guess
   we know it must be called and since it isn't flooding, it must
   finish. If it finishes its put is SR and that gets back to it being
   SR.  Do I need yet another lemma here?)}  and $\forall i < n,
 {call_i}$ is $SR$.

%(do I need to elaborate here? \Blue{YES, HOW ABOUT:
%  None of the  calls $call_i$ are flooding for $i<n$ and thus
%all start nodes  $start_i$  for $i<n$ are in SR}).

  By definition \ref{defn-wreachable} every node in the method
  containing $start_{n-1}$ is \emph{weakly-reachable} which includes
  the node $call_n$. Thus $call_n$ is \emph{weakly-reachable} and by
  definition \ref{defn-calls} is in ${\callsD}$.
}
\end{proof}

\subsection{Proof of Theorem \ref{thm-flooding}}
\begin{proof}
If there is some \emph{flooding} call, $call_n$, where $n \not\in {\callsD}-{\compsD}$ then either
\begin{description}
\item[a)] $n \not\in {\callsD}$, or
\item[b)] $n \in {\compsD}$.
\end{description}
By Lemma \ref{lemma-call-chain}, there must be a call chain starting on the 
cycle that leads to $call_n$ and by Lemma \ref{lemma-wr} $n \in {\callsD}$. 
%(Note: Need to correct of this being the first in a chain of flooded calls.)  
Therefore if the theorem does not hold, it must be because $n \in {\compsD}$ 
(and shouldn't be).

If $n \in {\compsD}$ then from definition \ref{defn-comps} and lemma
\ref{lemma-sr}, we know that either the method with label $n$ has
finished ($put_n$ or $get_n$ are SR), or the method with label $n$ is
partially executed directly in the cycle and none of the flow-paths
leading from the last node of the method in the cycle will suspend
before reaching a put.  Therefore $call_n$ cannot be flooding.
\end{proof}
%========================================

\subsection{Proof of Theorem \ref{theorem-main}}
\begin{proof}
%\noindent\textbf{Proof of theorem \ref{theorem-main}.}
Assume that there is a system execution $E$ of a given program with
flooding wrt.\ a method $m$.  We need to prove that flooding of a call
to $m$ is detected (or at least a call $m'$ such that flooding of $m'$
implies flooding of $m$).

Consider a system execution $E$ which is flooding  wrt.\ to method $m$.  Since
there are finitely many $m$ calls in the code, the execution must also
be flooding  wrt.\ one of these calls, say labeled $n$.  Thus it suffices
to consider each textual call rather than each method.  Let us assume
that $call_n$ is flooding in given execution $E$.
%%For  $E$  the call $call_n$
This call must be caused by a cycle $C$, repeated without bound.  (%In $E$ 
The cycle may be interleaved with other flooding or non-flooding
cycles.)  Thus the call $call_n$ must occur 
in the iteration of the cycle, or in 
a method called (directly or indirectly) from the iteration,
i.e., there must be a call path from the cycle to the call.
%the call  $n$  may be off-cycle).
Thus flooding wrt.\ $m$ in the sense of \ref{def-flooding-ex}  
implies flooding wrt.\ a cycle in the sense of \ref{flooding-cycle}\,.
The theorem therefore follows by theorem \ref{thm-flooding}\,.
%proved below. 
\end{proof}
%=======================================================================

\ignore{%=============================================================== 
  Before proving \ref{thm-flooding} we first reduce and simplify the
  set of executions without removing flooding of $m$.
  \begin{lemma}[Reduced execution set]
  \label{lemma-reduced-set}
  If there is an execution of a given program which is flooding wrt.\ a
  method $m$, our algorithm detects flooding of a call to $m$ or of a
  call $m'$ such that flooding of $m'$ implies flooding of $m$.
  \end{lemma}
  \noindent\textbf{Proof of lemma \ref{lemma-reduced-set}.}
  }%====================================================================


% LocalWords:  wrt
