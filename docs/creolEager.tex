\documentclass[12pt]{article}%[preprint],%a4paper,English
\usepackage{hyperref}
\usepackage{graphicx}
\usepackage{framed}
\usepackage[normalem]{ulem}

\pdfoutput=1

\input{preamble}

	
\renewcommand{\qedsymbol}{\rule{1ex}{1ex}}
\definecolor{shadecolor}{rgb}{0.9,0.9,0.9}
\newcommand{\Blue}[1] {\textcolor{blue}{#1}}%{\textsf{#1}}%
\input{definitions}    
\input{opsemDefs}
%
\newcommand{\knip}{\vspace{-1.5mm}}
\newcommand{\OLDFOOT}[1]{}%{\OO{TODO: #1}}
%\newcommand{\IFLONG}[1]{}%{\OO{TODO: #1}}
\newcommand{\TODO}[1]{}%{\OO{TODO: #1}}
\newcommand{\oo}{object-oriented\xspace}%{\textsf{#1}}%
\newcommand{\get}{get\xspace}%{\textsf{#1}}%
\renewcommand{\NEW}[1]{#1}%{\Blue{#1}}%{\textcolor{blue}{\textit {#1}}}%{\textsf{#1}}% {#1}
%\renewcommand{\RED}[1]{\textcolor{magenta}{#1}}
\newcommand{\callsE}{\ensuremath{\mathit{calls_C^{R}}}\xspace}% from Executions
\newcommand{\callsD}{\ensuremath{\mathit{calls_C^{S}}}\xspace}%  from Detection
\newcommand{\compsE}{\ensuremath{\mathit{comps_C^{R}}}\xspace}% from Executions
\newcommand{\compsD}{\ensuremath{\mathit{comps_C^{S}}}\xspace}%  from Detection
%\begin{frontmatter}

\begin{document}

\title{On Detecting Over-Eager Concurrency  \\
       in  Asynchronously Communicating  \\
       Concurrent Object Systems%
\thanks{This work was done in the context of the EU projects
   FP7-610582  \emph{Envisage: Engineering Virtualized Services}
    (\texttt{http://www.envisage-project.eu}) and
FP7-ICT-2013-X \emph{UpScale: From Inherent Concurrency to Massive 
Parallelism through Type-based Optimizations}
  (\texttt{http://www.upscale-project.eu}).}}

\author{Charlie McDowell and Olaf Owe\\
 \small{University of California, Santa Cruz, Dept.\ of Computer Science, USA, }\\
 \small{and University of Oslo, Dept.\  of Informatics Norway} 
 }
\date{\today}
\maketitle

\lstset{language=ABS}
\lstset{basicstyle=\ttfamily}
\ignore{
\lstset{backgroundcolor=\color{codebg}}
\lstset{frame=single}
\lstset{framesep=10pt}
\lstset{rulecolor=\color{codeframe}}
\lstset{upquote=true}

}

\lstset{emph={awk}, emphstyle=\textbf}

\section{Introduction}


Today, concurrency is a key aspect of the computer systems forming our
infra-structure. It is essential in distributed systems and net-based
service systems such as cloud computing, %, but also in multi-core systems
%applications.  
as well as % in % and also in %order to exploit 
multi-core computers.
\OO{Since  it is easier to reduce parallelism} than to increase the amount
of parallelism, 
it is a non-trivial 
challenge to design systems that 
allow %exhibit %exploit %work with
the desired amount of concurrency --  and in a correct manner.  In
practice many systems rely on centralized control or
synchronization of blocks of code to
make programs dealing with shared data work correctly, including
thread-based object-oriented concurrency, which is %form %are
the most common paradigm used to program distributed systems today.  
However, synchronization restricts parallelism
and slows down overall performance.  
%\sout{The thread-based concurrent model suffers from this.}
While synchronization 
primitives for notification/signaling may improve efficiency, they 
%Thread-based object-oriented concurrent programs suffers from this, and 
%in addition the synchronization primitives based on waiting and notification 
are difficult to use correctly because they break modular reasoning and
understanding.

The actor model has been acknowledged as a natural way of programming
concurrent systems, and is based on a simple semantics allowing
modular reasoning \cite{Hewitt77,Agha86,DAghaT04,218170}.  It has been
extended to the object-oriented setting in the form of active
concurrent objects, interacting by means of remote method calls.
Asynchronous methods increase efficiency by allowing non-blocking calls;
and shared futures enable even more efficient interaction, allowing
objects to share computation results without waiting for the results
\cite{Yonezawa86,Halstead85,liskov88pldi}. For instance a caller who
does not need the result of the called method may pass the future
identity of the result to other objects without (itself) waiting for the
result to appear.

Inter-object concurrency comes for free in the sense that each object
can run concurrently with other objects.  Intra-object synchronization
is handled in a modular manner without the use of external notification.
The concurrency model allows unrestricted concurrency with a
compositional semantics. Thus it enables %simplifies
efficient programming, class-wise understanding, verification, and testing.
However, this unrestricted concurrency model
does not come without a price.  This
programming style may give rise to deadlocks, and it is easy
%  Although not unique to programs created using Creol style synchronization,
% this programming style does make it quite easy 
to create programs that are class-wise semantically
correct but that fail due to over-eager creation of %suspended 
method calls. A system may feed an object with more calls than it is
able to handle, regardless of its processing speed.  We refer to this
situation as \emph{flooding} of the object.

While analysis of deadlock situations for this concurrency model has
been investigated in several ways, we are not aware of analysis of
object flooding for this concurrency model. In this paper we define
and exemplify the concept of flooding, distinguishing between strong
and weak flooding.  The scientific contribution of the paper is to
propose a static analysis method to detect possible flooding
situations, and prove its soundness (no false negatives).  Since 
static analysis of flooding cannot be both sound and complete,
detection of flooding may not imply a real flooding situation.
However, when no flooding is detected, this
%on of no flooding
 implies that there are no real flooding situations (soundness).

Something on related topics, like process management and process balancing...
data race detection.

We ignore flooding caused by direct recursion,
since this is not considered a concurrency problem.

\begin{figure}[t!]
$$
%
  \begin{array}{lrl@{\hspace{1em}}l} 
    \ifdecl & \bnfdef & 
    \kw{interface}\ I\ \bnfoptional{\kw{extends}\ \bnfplus{I}}^? 
%    \{\many{M_s}\} 
    \{S^*\} 
    & 
    \text{interface declaration} 
    \\ 
    \cldecl & \bnfdef & \kw{class}\ C([\cpofT]^*
    )\
    \bnfoptional{\kw{implements}\ \bnfplus{I}}\ \\ & & 
  \{[\wofT  \ [:=e]^?]^*
    \
    \bnfoptional{
    s
    %\CD{\{w := \new \ C(\elist{e})\}}
    }^?\
    %\many{M}
    M^*
\}&   \text{class definition} 
    \\  
%%%%      
%    \cldecl & \bnfdef & \kw{class}\ C([\cpofT]^*)\
%	%\bnfoptional{
%	\kw{implements}\ \bnfplus{I}
%	%}
%	\    
%    \{[\wofT \ [:=e]^?]^*\ M^*\}
%    &   \text{class definition} 
%    \\ 
%    M & \bnfdef & T\ m ([\xofT]^*)\ \{ [\kw{var}\ [\xofT]^*]^?\ s\;;\ \RETURN \ e\}
%    & \text{method definition} 
%    \\
    M & \bnfdef & 
    S\ B
	& \text{method definition} 
    \\
    S & \bnfdef & T\ m ([\xofT]^*
    )&  \text{method signature} 
    \\ 
    B & \bnfdef & \{[\kw{var}\ [\xofT \ [:=e]^?]^*;]^?% \many{\xofT}
    \ [s;]^?\ \PUT \ e\} & \text{method blocks} \\    
    T & \bnfdef & I %C
 \bnfbar \Int \bnfbar \Bool \bnfbar \String \bnfbar \Void \bnfbar \Fut{T}
    &  \text{types} \\
    v & \bnfdef & x \bnfbar w 
    & \text{variables (local or field)} 
    \\ 
    e & \bnfdef &  \kw{null} \bnfbar \this\ 
    \bnfbar v  \bnfbar cp  \bnfbar f(\elist{e}) &
    \text{pure expressions} 
    \\ 
    s & \bnfdef  & 
     v := e \bnfbar 
     v := v!m(\elist{e})  \bnfbar 
     v := [v.]^? m(\elist{e})         & \text{statements} 
 \\ &&  
    \bnfbar  [\AWAIT]^?   v := \GET \ e  \bnfbar \AWAIT\ e 
 \\ &&
    \bnfbar \kw{if}\ e\ \kw{then} \ s\ [\kw{else}\ s]^? \ \kw{fi} 
    \\ &&
    \bnfbar v := \new \ C(\elist{e})
    \bnfbar  \kw{skip} \bnfbar s;s  
\\
\end{array}
$$
\caption{\label{fig-creol} Core language syntax, with $C$ class name,
  \cp formal class parameter, $m$ method name, $w$ fields, $x$ method
  parameter or local variable, and where $\frv$ is a future variable.
  We let $[~]^*$, $[~]^+$ and $[~]^?$ denote repeated, repeated at least
  once and optional parts, 
  respectively, and $\elist{e}$ is a (possibly empty) expression list.
  Expressions $e$ and functions $f$ are side-effect free.}
\end{figure}



\section{A core language}

We consider a minimal core language inspired by Creol and its
successor ABS
\cite{Johnsen04b,johnsen07sosym,Johnsen05d,HATS1.2}. %Johnsen04e,
The Creol language
exploits the paradigm of concurrent, object-oriented,  active objects
using remote method call and  asynchronous method calls as the interaction
mechanism, adding support for non-blocking method calls to avoid
unnecessary waiting.  Shared-variables as well as 
 thread-based notification are
%replaced by a
avoided, introducing a
release mechanism, allowing an object to perform other tasks while
waiting for a condition to become true or for a method result to
appear.  Local data structures are captured by abstract data types
rather than (concurrent) objects.
% Modular semantics and modular understanding and reasoning are
% possible, and thus an object can be tested in isolation since its
% semantics is not changed by its environment.  
The resulting paradigm has a compositional semantics  and supports
modular reasoning and understanding of classes, and thus an object can
be tested in isolation since its semantics are not changed by its
environment \cite{Johnsen08,Din12jlap,din12sefm}.
\ignore{
Creol (and others) have suggested the use of concurrent objects
communicating via asynchronous method calls and futures, as a pathway
to better reasoning about concurrent systems. 

The communication and
synchronization model of Creol simplifies deadlock detection, allows
for ...

However, these advantages do not come without a price.  Although not unique to
programs created using Creol style synchronization, this programming
style does make it quite easy to create programs that are semantically
correct but that fail due to over eager creation of suspended method
calls.}


%The syntax of our core language is given in figure \ref{fig-creol}.


The core language presented in figure \ref{fig-creol}
includes standard
statements for assignment, \kw{skip}, conditionals, and sequential
composition.  We use a Java-like syntax, but use := for assignments.
%to not confuse 
%basic statements for first-class futures.
Methods are organized in interfaces and classes in a
standard manner.  A class $C$ takes a list of formal  parameters
$\classpar$, and defines fields $\Att$, optional initialization statements $\elist{s}$, and methods $\many{M}$.  
There is read-only access to 
%class parameters $\classpar$,
% method parameters $\many{x}$, and implicit variables, such as
 \this, referring to the current object, as well as %and 
the implicit method parameter \destiny, referring  to the future of the call
of the current method.
A method definition has the form $m (\elist{x})\{\kw{var }\elist{y};\
s;\ \PUT\ e\}$, when ignoring type information, where $\elist{x}$ is
the list of formal parameters, $\elist{y}$ is an optional list of
\emph{method-local variables}, $s$ is a sequence of statements, and
the final 
\emph{put} statement writes
the value of $e$ %s put 
in the future of the call, ``resolving the future''.
%
A future variable $\frv$  declared by $\Fut{\im{T}} \frv$  may refer to
%ndicating that $\frv$ may refer to 
futures  containing values  of type
$T$. The call statement $\frv := x!m(\elist{e})$ invokes the method $m$ on
object $x$ with input values $\elist{e}$.  The identity of the
generated future is assigned to $\frv$, and the calling process
continues execution without waiting for $\frv$ to become resolved.  The
query statement $v :=  \GET\ \frv$ is used to fetch the value of a
future. This statement blocks until $\frv$ is resolved, and then 
assigns the value contained in $\frv$ to $v$. 

The non-blocking release statement $\AWAIT \ v := \GET\ \frv$ suspends
the current process as long as $\frv$ is not resolved,
allowing other (enabled) process of the object to continue.
This gives rise to more efficient programming with futures.
Similarly, the statement
$\AWAIT \ c$ %, where $c$ is a 
suspends the current process as long as the Boolean condition $c$ is
not satisfied.
%
The remote call $v :=  x.m(\elist{e})$ blocks while waiting for
the future of the call to be resolved, and may be seen as a 
shorthand for  $\frv := x!m(\elist{e}); v :=  \GET\ \frv$ 
for a fresh $\frv$.
%Similarly 
The construct for local calls, $v:=m(\many{e})$,
employs a standard stack-based execution.

\ignore{
The non-blocking release statements,
$\AWAIT \ c$, where $c$ is a Boolean condition, and
%$\RELEASE$, 
$\AWAIT \ v :=  \GET\ \frv$, 
 release the  current process
%to be able to release the current process.
 as long as 
the condition is not satisfied or 
$\frv$ is not  resolved, respectively.
This gives rise to more efficient programming with futures.}


\ignore{Further statements for process control include a non-blocking
statement
$\AWAIT \ c$, where $c$ is a Boolean condition, and
%$\RELEASE$, 
$\AWAIT \ v :=  \GET\ \frv$, 
which releases the  current process
%to be able to release the current process.
 as long as $\frv$ is not yet resolved. 
This gives rise to more efficient programming with futures.}
%So a more flexible way to query a future will be $\RELEASE; v := \frvq$. 
%The discussion about process releasing statements can be found in
%\cite{din12jlap}. 
%we show how process release statements, including a 
%releasing query statement
% It is possible to add a treatment of 
% %the discussion about 
% process release statements 
% as a straight forward extension of the present work,
% see section 
%following the approach of \cite{din12jlap}.
%without inte violating the other part of the paper.
% where the issues related to futures are the main concern. 
%However, %we choose to focus on a  general  set of \ABS 
%
Object variables are typed by interfaces, 
and remote field access is (syntactically) forbidden.
We assume that call and query statements are
well-typed.  
\ignore{If $x$ refers to an object where $m$ is defined with 
input types $\many{S}$
% no input values 
and return type $T$, the following code is well-typed
when $\many{e}$ is of type  $\many{S}$:
$\Fut{T} \frv;\ T\ v;\ \frv:=x!m(\many{e});\ v:=\frvq$,
which represents 
a traditional synchronous and blocking method call.
This call is abbreviated by the notation $$v:=x.m(\many{e})$$
%In the example below
%We use the notation $$v:=x.m()$$ to abbreviate this call.
Note that the call $v:=\this.m(\many{e})$ will block,
and thus a construct for local calls, say  $v:=m(\many{e})$,
would be useful.}
% Class instances are concurrent, encapsulating their own state and
% processor.  Each method invoked on the object leads to a new process,
% and at most one process is executing on an object at a time.  Object
% communication is \emph{asynchronous}, as there is no explicit transfer
% of control between the caller and the callee. 
We here ignore inheritance since it
does not influence our discussion on flooding.



\section{Flooding Cycles}

As a motivating example we will
consider  versions of the publish/subscribe example.
%taken from \cite{din14jlap}.
Clients may  \lstinline{subscribe} to a service object
and the service object will ensure 
that subscribing objects receive information about ``news''.
Clients are notified by news by 
%The interface of Clients contains
 the   \lstinline{signal} method.
%for this purpose. 
%Clints call \lstinline{subscribe} on the 
The service object is using a number of proxies to 
handle all the clients and is using  an underlying news producer
to obtain news. The service object is using futures to avoid
being delayed by waiting for news to be available,
thus it may continuously respond to clients.
The  interfaces of these units are given 
in figure \ref{example-subscr-interfaces}.

\begin{figure}
%Example
\begin{abs}
data News=E1|E2|E3|E4|E5|None;    $\hfill$ // example of different news
interface ServiceI{
  Void subscribe(ClientI cl);     $\hfill$ // called by Clients
  Void produce()}                 $\hfill$ // called by Proxies
interface ProxyI{
  ProxyI add(ClientI cl);         $\hfill$ // called by Service objects
  Void publish(Fut<News> fut)}    $\hfill$ // called by Service objects
interface ProducerI{
  News detectNews()}              $\hfill$ // called by Service objects
interface NewsProducerI{
  Void add(News ns);              $\hfill$ // called by main when news arrive
  News getNews();                 $\hfill$ // called by Producer objects
//  List<News>getRequests()}     // not used !
interface ClientI{
  Void signal(News ns)}           $\hfill$ // called by Proxies
\end{abs}
\caption{\label{example-subscr-interfaces}
The interfaces to complete the subscription example.}
\end{figure}
 %=========================================

A high-level Creol implementation of 
%the example of
 the publish/subscribe model is given in 
figure \ref{example-subscr} and is taken from Din and Owe\cite{din14jlap}.


\begin{figure}
%Example
\begin{abs}
class Service(Intlimit,NewsProducerInp) implements ServiceI{
  ProducerI prod;ProxyIproxy;ProxyIlastProxy;
  { prod := new Producer(np); 
    proxy:= new Proxy(limit,this);lastProxy:=proxy;this!produce()}
  Void subscribe(ClientIcl){lastProxy:=lastProxy.add(cl)}
  Void produce(){var Fut<News>fut:=prod!detectNews();proxy!publish(fut)}}

class Proxy(Intlimit,ServiceIs) implements ProxyI{
  List<ClientI> myClients:=Nil;ProxyInextProxy;
  ProxyI add(ClientIcl){
    var ProxyI lastProxy=this;
    if length(myClients)<limit then myClients:=appendright(myClients,cl)
    else if nextProxy==null then nextProxy:= new Proxy(limit,s) fi;
             lastProxy:=nextProxy.add(cl) fi;
    put lastProxy}
  Void publish(Fut<News>fut){
    var News ns=None;
    ns =get fut; myClients!signal(ns);
    if nextProxy==null then s!produce() else nextProxy!publish(fut) fi}}

class Producer(NewsProducerI np) implements ProducerI{
  News detectNews(){
    News news:=None;
    news:=np.getNews(); put news}}
class NewsProducer implements NewsProducerI{
  List<News>requests:=Nil;
  Void add(News ns){requests:=appendright(requests,ns)}
  News getNews(){
    var News firstNews:=None; await requests /= Nil;
    firstNews := head(requests);requests:=tail(requests); put firstNews}
 }

class Client implements ClientI{
  Newsnews:=None;
  Void signal(News ns){news:=ns}}

\end{abs}
\caption{\label{example-subscr}
A simple subscription example.}
\end{figure}
 %=========================================

Modifying Client and Proxy as shown in figure \ref{example-subscr-flooding}, results in a program that will flood the system with suspended calls. 
The changes are at lines 9 and 14 in the revised code.
The change is to shift requiring the actual news to have arrived from the Proxy (\lstinline{ns=get fut; myClients!signal(ns);})
to the Client (\lstinline{news:=get fut}).


\begin{figure}
%Example
\begin{abs}
class Proxy(Intlimit,ServiceIs) implements ProxyI{
  List<ClientI> myClients:=Nil;ProxyInextProxy;
  ProxyIadd(ClientIcl){
    var ProxyIlastProxy=this;
    if length(myClients)<limit then myClients:=appendright(myClients,cl)
    else if nextProxy==null then nextProxy:= new Proxy(limit,s) fi;
    lastProxy:=nextProxy.add(cl) fi; put lastProxy}
  Void publish(Fut<News>fut){
    myClients!signal(fut);
    if nextProxy==null then s!produce() else nextProxy!publish(fut) fi}}

class Client implements ClientI{
  News news:=None;
  Void signal(Fut<News> fut){news:=get fut}}

\end{abs}
\caption{\label{example-subscr-flooding}
A flooding variation of the subscription example.}
\end{figure}
 %=========================================

This seemingly minor change, and one that would even seem to make sense in the interest of maximizing concurrency, is in
fact ``too much.'' We might naively take it even one step further and have the client instead do 
\lstinline{news:=await(fut)}, 
which has the additional advantage of allowing the Client to process the news items as they become available, rather then in
the order that the futures were received. In either case, the following sequence of calls can occur, which constitute a
flooding cycle (see definition \ref{flooding-cycle}).
\\

{\small
\samepage
%\begin{center}\begin{flushleft}%{verbatim}
\indent\lstinline{Service.produce} asynchronously calls \lstinline{Producer.detectNews}
\\
\indent\lstinline{Service.produce} asynchronously calls \lstinline{Proxy.publish}
\\
\indent\lstinline{Proxy.publish} asynchronously calls \lstinline{Client.signal}
\\
\indent\lstinline{Proxy.publish} asynchronously calls \lstinline{Service.produce}
%\end{flushleft}\end{center}%{verbatim}
}\\

\noindent
Each pass around this cycle, the asynchronous call to \lstinline{Proxy.publish} is processed as part of the cycle (step 3).
However, each pass around this cycle also spawns an asynchronous call to  \lstinline{Producer.detectNews} that is not processed as part of this cycle,
nor is there any attempt to synchronize this cycle with the completion of those calls to  \lstinline{Producer.detectNews}.
Depending upon the speed of
execution of the code along the path of the cycle, such a cycle can create an unbounded number of suspended calls to
\lstinline{Producer.detectNews}.

We call such sequences flooding-cycles. In this paper we present an algorithm to statically identify
programs that contain flooding-cycles. This approach is conservative in that if it reports that a program is free from
flooding-cycles then it is indeed free of such cycles, however, it may report flooding-cycles that are in fact bounded by
program logic, not amenable to static analysis. It will also report flooding-cycles that do not in practice produce an
increasing number of unprocessed calls due to the execution speed of the flooding-cycle. 

A flooding-cycle must always be racing against one or more specific asynchronous calls, either trivial calls (no waiting on
any future from the call) or calls where the resulting future is not read in the cycle, although the future may be passed
to a separate asynchronous call where it is read.  

The version of the program in figure \ref{example-subscr} has a flooding-cycle (cycle B in figure \ref{graph-orig}) 
that is racing against the \lstinline{Client.signal} calls
generated in \lstinline{Proxy.publish}. This will not cause a problem, provided the Client objects are able to process these signal
calls at least as fast as they are being generated. Our algorithm will alert the programmer to this situation, and the
programmer can determine if there is a real problem here, possibly with the aid of some additional program
instrumentation.  

In this case, the flooding-cycle will not create a flood for two reasons that are beyond the ability of our current approach to detect.
First, this cycle is in fact bounded by program logic. The cycle is walking down a finite chain of Proxy objects.
Second, after a finite number of times around the cycle the code will branch and make a pass around the cycle that includes the 
\lstinline{get}
(cycle A in figure \ref{graph-orig})
which results in a delay for more news to actually arrive, thus limiting the speed of cycle A.

The version of the program in figure \ref{example-subscr-flooding}
results in the graph shown in figure \ref{graph-modified}. This second graph is the same as that in figure \ref{graph-orig}
except that the \lstinline{get} node is now in the  \lstinline{Client.signal}
method (not shown in the graph) and there is no \lstinline{get} in cycle A. 
The result is that cycle A in the modified program
is a flooding cycle that is racing against both the
\lstinline{Client.signal} calls and also against the
\lstinline{Producer.detectNews} calls.  This is because rather than
wait in \lstinline{Proxy.publish} for the news to actually be produced
as in the first version (\lstinline{ns=get fut;}),
\lstinline{Proxy.publish} instead simply passes the future out to
another asynchronous call (\lstinline{myClients!signal(fut);}),
eliminating any progress coordination between cycle A and the
\lstinline{Client.signal} calls. Cycle A in this version of the program is more likely to be a
problem because it is not dependent simply on execution speed of some
code but is dependent upon the arrival rate of news items and in
practice will always result in the number of unprocessed calls quickly
growing to system limits.

\section{Identifying Flooding Cycles}

\begin{definition}[Flooding Cycle]\label{flooding-cycle}
A \emph{flooding cycle} is an execution cycle 
containing a call statement
%involving one or more asynchronous calls
%such that for at least one of those calls,  call it $o.m()$ occurring
 at a given program location, such that 
this statement may produce, directly or indirectly, an unbounded number of 
uncompleted
calls to a specific method.
%such that also an unbounded number of those calls are not completed.
%before any of those calls  has been completed.%
\ignore{COMMENT: well some might complete, as long as the number of uncompleted
ones is also unbounded.}
\end{definition}%

Flooding of this kind may depend on the relative
speed of different objects, and  may be difficult to detect even
when testing complete systems, due to their inherent
non-deterministic nature.  They may show up for instance at times when
the system load is high.

Flooding cycles that depend on unfair scheduling
may be avoided with fair scheduling of concurrent activity and
fair scheduling of  tasks inside one unit.
Flooding cycles that exist even with fair scheduling
are more serious, indicating bad programming.
We call such flooding \emph{strong flooding}.
%inherent, serious

\ignore{
%\section{Definition of Strong Flooding Cycles} %p
\begin{definition}[Strong Flooding Cycle]
\label{strong-flooding-cycle}
A \emph{strong flooding-cycle} is an execution cycle 
involving one or more asynchronous calls
such that for at least one of those calls, call it $o.m()$,
an unbounded number of the calls to  $o.m()$ may be produced by the cycle
before any invocation of  $o.m()$ has been completed,
even with fair scheduling between concurrent units and 
(enabled) tasks within each unit.
\end{definition}}

The top-level view of an algorithm to detect flooding-cycles is shown in figure \ref{flow-analysis}.
We first create a control flow graph (cfg) for each method where the nodes are method start nodes, method calls, \emph{get} statements, 
\emph{await} statements,
or \emph{put} statements (including implicit \emph{put} statements at the end of Void methods). %\footnote
Synchronous method calls will be represented as an asynchronous method call followed immediately by a \emph{get}.
All other statements are ignored. 
%In addition, the nodes are of two types,
%synchronous, or asynchronous. Synchronous nodes are \emph{get} statements or \emph{await} statements.

\begin{figure}
\begin{shaded}
\begin{enumerate}
\item Build the individual control flow graphs for each method including a start node, and a node for each \emph{call}, \emph{get}, \emph{await}, 
and \emph{put} statements.
\item Add call edges from call nodes to the start node of a copy of the corresponding method cfg, unless the call is recursive, in which case
create a call edge to the existing start node. Assign each call node a unique label.
\item Identify any cycles in the graph.
\item Use flow analysis to compute the \emph{put} edges from \emph{put}s to \emph{get}/\emph{await}. (What is going to happen with more complex examples
where there are multiple \emph{put}s that might reach a given \emph{get}?)
\item Compute the calls and comps sets for each cycle to identify flooding-cycles using algorithm xx.
\end{enumerate}\end{shaded}%
\caption{\label{flow-analysis}
Top Level Algorithm for Detecting Flooding-Cycles}
\end{figure}

In step 2 we connect the cfgs for each method with call edges, replicating the method's cfg for each non-recursive call so that we can
associate a specific call with a specific start/get/put. Note that we do not distinguish between multiple calls to the same method in one
object, and one call to a particular method in each of multiple instances of the same class. 
\footnote{Seems like maybe a place for a lemma that asserts that doing so won't result in false negatives.}

In step 3 we identify any cycles in the graph. Note, cycles can include call edges and flow edges.

In step 4 we add edges from \emph{put} statements to \emph{get} or \emph{await} statements that block on the value for that future.
\footnote{If we have the ambition of ensuring that 
there is no flooding when the algorithm finds none,
we must ensure that a ``get'' actually waits for the future
(``must instead of ``might'').}

In step 5 we apply the algorithm in figure \ref{reachable-analysis} to identify any flooding cycles.
If there is a cycle that creates futures (including implicit Void futures for trivial calls) that are not read
in the cycle\footnote{Not really in the cycle, but in the reachable nodes starting from the cycle}, 
then there is a flooding-cycle with respect to the call that produced the unread future. 
%(I guess this is where we introduce a theorem.)


\begin{figure}
\begin{shaded}
\begin{enumerate}
\item Starting with the graph $G$ from step 4 in figure \ref{flow-analysis}, mark all nodes \emph{not reachable}.
\item \label{recompute} Starting with the entry point to the cycle, do a depth-first traversal of $G$ and apply definitions
\ref{defn-sreachable} and \ref{defn-wreachable} to mark nodes as \emph{strongly-reachable}, \emph{weakly-reachable}, or neither.
\item Apply definitions \ref{defn-calls} and \ref{defn-comps} to compute the $calls$ and $comps$ sets.
\item If the previous two steps results in any changes to \emph{strongly-reachable}, \emph{weakly-reachable}, $calls$, or $comps$, 
go to step \ref{recompute}.
\end{enumerate}
\end{shaded}
\caption{\label{reachable-analysis}
Algorithm to compute the \emph{reachable} nodes in a graph relative to a given cycle.}
\end{figure}

%=======================================================
\subsection{Computing the call and comp sets (step 5)}
 We consider one cycle (at a time),
ignoring any external call initiating the cycle.

\noindent\textbf{Assumptions:}

\begin{itemize}
%\item

%\item
%If the same method is called several times from within the cycle,
%we include a copy of its body for each call.
%Thus each copy of a  method body has  at most  one caller from within the cycle.
\item
We assume a finite graph (even after expansion).
\item
The calls in the graph are labeled, 
say numbered by their textual occurrence in the (sub)program.
\item 
\Blue{Every method begins with a \emph{start} node and ends with a \emph{put} node, and those two nodes
will have the same label as the caller. Because of graph expansion, the caller will be unique except
in the case of a recursive call. In that case, the label is the label of the non-recursive caller except when
the cycle being considered includes the recursive call edge.
When the recursive call edge is part of the cycle the label will be that of the recursive caller.}
\item
\Blue{We mark each \emph{get} node with the set of labels corresponding to any \emph{call} nodes that may have
generated the future that reaches the \emph{get}.}
\item
Blocking calls are split into an asynchronous call $n$ and a (matching) \emph{get}
marked with  $n$.
\item
We treat blocking self-calls specially,
by not including an edge from the call node to the get node,
but instead  in-lining  the copy of the method
 (with an arrow to its
start node, and an arrow from each  \emph{put} node
back to the next statement). \Blue{Why do we need this?}
\item We treat \emph{await(future)} the same as a \emph{get}.
\item
Boolean \emph{await} statements are represented as nodes in the graph
(since they may affect method completion). They are assumed to never block when in a cycle and to
always block when not in a cycle.

\end{itemize}
\textbf{Analysis:}
The analysis considers one cycle (at a time)
together with surrounding parts of the program graph.
The analysis computes two sets; the set of calls that could possibly have occurred during 
the cycle or outside of the cycle when \emph{reachable} from the cycle without suspensions or blocking,
and the set of calls that must have completed during
the cycle or outside of the cycle when \emph{reachable} from the cycle without suspensions or blocking.
These sets are denoted $calls$ and $comps$, respectively. An algorithm to compute those sets is shown in
figure \ref{reachable-analysis}.

\ignore{
The analysis considers one cycle (at a time)
together with surrounding parts of the program graph.
The analysis will collect sets of calls $n$ from the cycle and call completions,
in  the cycle, or even outside  the cycle when \emph{reachable} from the cycle without suspensions or blocking,
or  when a predecessor of  a \emph{get} is related to a call in the cycle (NOTE: I don't know what this means.).
The analysis, uses two sets, denoted $calls$ and $comps$. An algorithm to compute those sets is shown in
figure \ref{reachable-analysis}.
}

\begin{definition}[Program Graph]\label{def-program-graph}
A \emph{program-graph} $G$ is a graph comprised of \emph{start}, \emph{call}, \emph{get}, \emph{await}, and \emph{put} nodes,
and \emph{flow}, \emph{call}, and \emph{future} edges, constructed according to steps 1, 2, and 4 of figure \ref{flow-analysis}.
\end{definition}

\begin{definition}[Flow Reachable]\label{def-flow-reachable}
Node $N_j$ is \emph{flow-reachable} from $N_i$ if there is a flow-edge $N_i \rightarrow N_j$. 
\end{definition}

\begin{definition}[Flow Path]\label{def-flow-path}
A \emph{flow-path} is a path made up of only flow-edges.
\end{definition}

\begin{definition}[Strongly Reachable]\label{defn-sreachable}
Node $N_j$ is \emph{strongly-reachable} with respect to cycle $C$ in graph $G$ if
\begin{enumerate}
\item \label{sr-incycle} $N_j \in C$, or
\item $N_j$ is not a \emph{get} node \Blue{or a boolean \emph{await} node}, $N_j$ is flow-reachable from $N_i$, $N_i$ is \emph{strongly-reachable}, and 
no other node $N_k$ is flow-reachable from $N_i$, or
% next rule used in EX14 (quasi example 2)
\item $N_j$ is a \emph{get} node with label set $s$, containing just one label, $n$;
the \emph{put} node with label $n$ is \emph{strongly-reachable};
$N_j$ is flow-reachable from $N_i$; $N_i$ is \emph{strongly-reachable};
and no other node $N_k$ is flow-reachable from $N_i$; or
\item \label{sr-start} $n \in {comps}$, $N_j$ is a start node with label $n$, $\exists$ a call edge $N_i \rightarrow N_j$, and $N_i$ is \emph{strongly-reachable} or

%possible additions
\item $N_j$ is a \emph{put} node with label $n$,
$N_k$ is a \emph{strongly-reachable} \emph{get} node with a label set containing $n$, and the node $call_n$ is \emph{strongly-reachable}.

\item $N_k$ is flow-reachable from $N_j$, $N_k$ is not flow-reachable from any  $N_i$ $i \not = j$, and $N_k$ is \emph{strongly-reachable}.% see Ex9
\end{enumerate}
\end{definition}


\begin{definition}[Weakly Reachable]
\label{defn-wreachable}
Node $N_j$ is \emph{weakly-reachable} with respect to cycle $C$ in graph $G$ if
\begin{enumerate}
\item $N_j$ is \emph{strongly-reachable}, or
\item \label{wr-start} $N_i$ is a \Blue{\emph{weakly-reachable}} start node and there is a flow-path from $N_i$ to $N_j$, or
%\item \sout{$N_i \rightarrow N_k$ is a call edge in $C$ and there is a flow-path from $N_i$ to $N_j$.}

\item\label{wr-calledge} 
\sout{$n \in {comps}$,} $N_j$ is a start node \sout{with label $n$}, $\exists$ a call edge $N_i \rightarrow N_j$, and $N_i$ is \emph{weakly-reachable} or

\item \sout{$N_j$ is a \emph{put} node with label $n$,
$N_k$ is a 
\emph{weakly-reachable} \Blue{(or should that be \emph{strongly-reachable})}
\emph{get} node with label $n$, and the node $call_n$ is \emph{weakly-reachable}.}


%\item $N_j$ is not a \emph{get} node, $N_j$ is flow-reachable from $N_i$, $N_i$ is \emph{weakly-reachable}, and 
%there is no node $N_k$ in $C$ that is flow-reachable from $N_i$, or % don't follow other flow edges for flow edges in the cycle
%\item $N_j$ is a \emph{get} node, all flow-predecessors of $N_j$ are \emph{weakly-reachable}. % no requirement on the put predecessors

\end{enumerate}
\end{definition}

\begin{definition}[Weakly Reachable Calls]
\label{defn-calls}
For cycle $C$, \callsD %\emph{calls(C)} 
is the set of labels of all \emph{weakly-reachable} call-nodes in $C$.
\end{definition}

\begin{definition}[Strongly Reachable Completions]
\label{defn-comps}
For cycle $C$, \compsD %\emph{comps(C)}
 is the set of labels ${n}$ such that 
\begin{itemize}
\item ${n}$ is the label for a \emph{put} node that is \emph{strongly-reachable}, or
\item ${n}$ \Blue{is a member of the set of labels} for a \emph{get} node that is \emph{strongly-reachable}, or 
\item $N_i$ is a \emph{call} node and the head of a call-edge in $C$; 
for every node $N_j$ on a flow-path from $N_i$ to a \emph{put} node,
$N_j$ is not a Boolean \emph{await} node, if $N_j$ is a \emph{get} node then $N_j$ is \emph{strongly-reachable}; 
and $n$ is the label of those \emph{put} nodes.
%Note - I dropped the requirement that the nodes on the paths be WR because that is implied because start_n must be in the cycle making
% it SR and by WR.2 all of the nodes in question will be at least WR.
\end{itemize}
\end{definition}



\begin{theorem}[Flooding with respect to Cycle]
\label{thm-flooding}
If $call_n$ is \emph{flooding} with respect to cycle $C$ then 
flooding is detected by our algorithm, i.e., $n\in (\callsD -\compsD)$.
\end{theorem}


\subsection{Applying the Algorithm to the example program}


\begin{figure}[t!]
\includegraphics[width=15cm]{graphOrig}
\caption{\label{graph-orig}
The graph and call/comp sets for the original version of the program (figure \ref{example-subscr}).
}
\end{figure}

\begin{figure}[t]
\includegraphics[width=15cm]{graphModified}
\caption{\label{graph-modified}
The graph and call/comp sets for the modified version of the program (figure \ref{example-subscr-flooding}).
}
\end{figure}


Figures \ref{graph-orig} and \ref{graph-modified} show the call and comp sets
for the two versions of the producer consumer problem above. To conserve space, all method names are abbreviated to the
first letter of the class and the first letter of the method except that we use X for the class Proxy to further disambiguate it from Producer. 
For example, Producer.detectNews is Pd and Proxy.publish is written Xp.

In figure \ref{graph-orig} we see there are two cycles involving only flow-edges and call-edges.
The call to Client.signal (Cs) is being flooded by both cycles.
This does not produce an actual flood because the amount of work required by the
Client to complete a signal call is trivial and thus the Client objects easily keep up with the calls. 
Also, the rate of execution for cycle A is
limited by the actual arrival of news items from the NewsProducer (\lstinline{await requests /= Nil;}), 
which further limits the rate at which this cycle generates asynchronous calls to CLient.signal.
Also, although not observed by our algorithm, cycle B is in fact finite, as it walks down the chain of Proxies.
We call this \emph{weak flooding}, i.e. flooding that is harmless, given underlying fairness of concurrent objects and
fair scheduling of tasks within an object.

In the modified version of the program as reflected in figure \ref{graph-modified} there is an additional flood of
Pd (Producer.detectNews) by both cycles. This flooding-cycle is serious, and in practice will flood the system almost immediately.
Unlike in version 1, there is no \emph{get} regulating the speed at which cycle A cycles. Furthermore, the Pd calls are dependent upon the
arrival of news items not simply limited by processor execution speed, as indicated by the presence of the boolean \emph{await} in Ng (News.getNews).
We call such flooding \emph{strong-flooding}.

\ignore{
\subsection{Weak and strong flooding}
When a cycle floods with respect to a call $o.m$,
we say that we have \emph{strong flooding} 
when the method contains an  await or get node
(not resolved in the context of the cycle),
otherwise \emph{weak flooding}.
Strong flooding indicates a serious flooding case,
while weak flooding typically indicates flooding 
that is harmless, given underlying fairness of concurrent objects
and  fair scheduling of tasks within an object. 
%Weak flooding of several calls to the same object could cause serious flooding.

The first version of the subscription example
has weak flooding of client calls (\emph{signal}),
whereas the second has strong flooding both with respect to
to client calls and  to \emph{detectNews} calls.
}
%\newpage
%\newpage


%==============================================================

\section{Main Theorem (no false negatives)}

\ignore{Our definitions try to approximate infinite executions with 
  sets based on maps to finite executions. 
This makes the concept of unbounded more direct.}

%\textbf{Executions.}
In order to formally state properties of detection of flooding in
executions, 
%In order to state and prove properties of flooding,
we need some understanding of \emph{executions} and execution order, 
in particular
how method calls and completions are handled.
To capture independent processor speeds,
we  give a partial order semantics.
In order to define unboundedness in a natural manner,
we  approximate infinite executions with 
infinite  sets based on maps to finite executions. 
Other aspects such as state values,  evaluation,
and sequential execution steps are less central
and are not specified in detail. 

%At any point in an exection, the 

In our concurrency model,
each object is responsible for executing methods called on that object.
Each method instance corresponds to a \emph{process} at run time, which may be suspended by \emph{await} statements.
\OO{A started process continues until the end (given by a $put$),
 %or to an internal loop 
 to a suspension point, %(i.e., an await% with a false condition)
or to a get where the future value is not available.}
Suspended processes are kept in a queue together with incoming calls.
In the semantics below, the queues are implicitly given by 
looking at the (past)  execution history.
%\footnote{Could also stop after a satisfied await?}
An object can only execute one process at a time.
When a process is suspended, the object may continue
with another enabled process (if any).
\ignore{NOT NEEDED? We here assume that (enabled) suspended processes 
have priority over new incoming calls.}%
Local synchronous calls are done as usual, in a stack-based manner.
Local asynchronous calls are queued together with incoming calls.
%
Delays in execution are caused by delayed start of a method
execution, say when the processing object is busy,  
by  suspension, % (by await or suspending get), 
and by blocking \emph{get}s.
More formally we define system and object executions below.
\ignore{
   NEEDED?? For simplicity, we assume that resumption of suspending
   \emph{get}s  (await on future) are not unnecessarily delayed when the
   future is available; thus we do not distinguish blocking and
   suspended \emph{get}s   wrt.\ delays, both are assumed to continue without
   delay when the future value is present.}

%\subsection{Short Version}

\begin{definition}[System Execution and Object Execution]\label{def-ex}

A \emph{system execution} $E$  is a mapping from time to 
system execution at that time.
A \emph{system execution at time $t$}, denoted $E[t]$, is given by two partial mappings:
\begin{itemize} \item 
  a partial mapping from object identities to
  \emph{object executions} (one for each object in the system) at time $t$,
  describing the object execution states %/steps  
up to time $t$.
  $E[t][oid]$ denotes the execution of object $oid$ at time $t$.
%together with a set of (shared) futures, given by 
\item
  a partial mapping of future identities to values, reflecting all
  generated future values up to time $t$.
   $E[t][\mathit{fid}]$ denotes the %future 
value of $\mathit{fid}$ at time $t$. 
\end{itemize}
%\begin{definition}[Object Execution]\label{def-ex}
%
The \emph{execution of a given object %of given class 
at a given point in time} may be seen as a finite sequence of 
execution states, where
%, infinite if the execution does not terminate. 
% Each state is indexed by the corresponding node (label)
% in the program (graph), and
each state includes relevant information such as 
%future identities and 
object state and local state, as well as the program counter value
(and label in the program graph), and the unique future value of the
invocation of the each % process 
method execution.
Generated object and future identities are globally unique.
The execution %of a method 
must follow the code of the method of the class of the object,
letting the execution of a \emph{put} or an \emph{await}
take the object to an \emph{idle} state.
After an idle state the object may start a new method
execution or continue an old %suspended
 execution, after the last \emph{await},
%a release point
if the \emph{await} condition is enabled.
%
%Each object must progress with time. Thus for a system execution $E$ we have
%Thus the continuation of an idle state is is general non-determinitic.
%Each execution step must be legal, in goes ....
\end{definition}
Thus the only  non-deterministic steps appear after idle states
and at branches that depend (directly or indirectly) on the value resulting from a  \emph{get}, which depends on the environment.
%We may assume some general properties about executions%We may assume that execution continues with time
%until no further execution step is possible by any object.
We may  assume that  executions  obey the following  general properties.
\begin{description}%===================================
\item[Object Progress:]
A method execution must continue (with time)
as long as it has enabled steps.
% If the execution of  an object has an enabled step at time $t$,
%   the object must eventually continue to execute.
\ignore{
If the execution of an object at a given time ends in a non-idle state,
the process will eventually continue or end in a \emph{get} that is 
blocked at all later times.}

\ignore{is not idle
  If the execution of  an object has an enabled step at time $t$,
   each object continues to execute if possible
  %(\emph{object progress}):
  An object execution must eventually continue as long as there are 
  enabled steps (i.e., the object is idle and there are enabled processes,
  or the object is not blocked in a \emph{get} in a method execution).%
  \footnote{More formally: If an object $o$ stops at time $t$, i.e.,
  $\forall t' \,.\ t'<t \Rightarrow E[t'][o]= E[t][o]$, it must be
  blocked in a \emph{get}, and the future value is never generated, 
  or it is in an idle state with no enabled process.}
  Thus  each enabled step will be done by an object unless the object 
  gets an unbounded number of processes, in which case
  it will not be able to perform all of them.
}
\item[Historic Monotonicity:]
Executions must be \emph{historically monotonic}
in the sense that a system execution $E$
seen at two different times,  $t1$
and  $t2$ such that $t1<t2$,
must agree on the past, i.e.,  $E$ must  satisfy
% $$E[t1]\leq E[t2]$$ where $s\leq s'$ means that 
$E[t1][\mathit{fid}] \not=\bot  \Rightarrow E[t1][\mathit{fid}] =E[t2][\mathit{fid}]$
 for each future identity $\mathit{fid}$,
and
$E[t1][oid] \not=\bot  \Rightarrow E[t1][oid] \leq E[t2][oid]$ 
for each object identity $oid$ where $\leq$ denotes 
the sequence prefix ordering.
Furthermore, %Moreover,
 $E[t][oid]$ is defined if and only if
the object $oid$ has been  created in the  execution  $E[t][oid']$ of
some object $oid'$ by a  \textbf{new} statement. % Do we need to specify that the new happened ``before''?
Similarly,  $E[t][\mathit{fid}]$ is defined
if and only if
the future  $\mathit{fid}$ has been resolved in  $E[t][oid']$ by some object  $oid'$.

\item[Partial Ordering:]%
%[Partially Ordered by the ``after''-Relation]
Each system execution  must satisfy a partial ordering 
(called \textbf{after}) relating the different object executions  as
follows: 
%Each object execution %of an object $o$ (of class $C$) 
%must satisfy
\begin{itemize}
\item The first state of an object of class $C$
(other than the initial object)
 %must %be the beginning of its initiali
must occur \emph{after}
%NOTE during if synchronized ....
the object has been generated by a new $C$ construct
and must be the start of its initialization part (with an object state
reflecting class parameters). 
The object  mapping is extended
by associating the new object identity with this object execution.
\item The start of a method execution by an object $o$ must appear \emph{after} 
      the corresponding call.
\ignore{\item the next step of an execution of $o$ must  be the next statement
      in the branch of the active process (evaluating if tests and following the
      appropriate branch), unless it is an await or get statement, or local call.}
\item A blocking \emph{get} must be executed \emph{after} 
      the future value has been generated.
\item 
%a state representing the next statement 
%the state   representing the state \emph{after}
The execution of  an \emph{await} of a process $p$
must occur \emph{after} the await is enabled 
%(and when it is the next  state of the process $p$) 
and may only appear after an idle state.\footnote{Do we need to worry here about awaits that are disabled again before they run?}
\ignore{
    if the next statement is an await the remaining process (including
   the await and and the local state) is placed on the process queue of the
   object, allowing the object to proceed with another process on the
   queue, provided it is enabled.}
\item When a \emph{put} is executed (finishing the current process) 
% the current process is said to be   inished and
the future  mapping is extended by associating the produced
value with the  future identity of the current process (and we say that the future value has been
  generated). The object is in \emph{idle} state.
%\item The execution  of objects not generated (apart from the initial object)  %is $\bot$, and the future value of futures not (yet) produced is $\bot$.
\end{itemize}
\end{description}
The last two properties  may be proved by induction from a detailed operational semantics
such as that from Din et al.\cite{Din12jlap}.

\begin{definition}[Unbounded Number of Calls]\label{def-unbounded}
  A system execution is said to produce \emph{an unbounded number of
    calls} (of a certain kind) if for every bound there is a time $t$
  such that the system execution at time $t$ has a higher number of
  calls (of the given kind) than the bound.
\end{definition}

\begin{definition}[Flooding]\label{def-flooding-ex}
  An execution is flooding if it produces an unbounded number of
  uncompleted method calls.  It is flooding wrt.\ a method $m$ if it
  produces an unbounded number of uncompleted method calls to $m$.
\end{definition}


The theorem below expresses that our algorithm covers all possible floodings
(no false negatives).

\begin{theorem}[No false negatives]
\label{theorem-main}
If there is an execution of a given program which is flooding wrt.\ a
method $m$, our algorithm detects flooding of a call to $m$ or of a
call $m'$ such that flooding of $m'$ implies flooding of $m$.
\end{theorem}
\section{Proofs}

%\ignore{ move??
\textbf{Notation} of nodes: we write $call_n, put_n$, $start_n$
   where $n$ is the label of the corresponding call, and
   $get_s$  where  $s$ is a single label or a label set.
   %Similarly start nodes are indexed by the label of the corresponding call.}


In order to prove  theorem \ref{thm-flooding}
we first define a notion of execution tree, and then prove that flooding
wrt.\ a given cycle and method call 
%implies that there are uncompleted calls in the execution trees.
%We prove below that flooding according to definition   \ref{flooding-cycle} 
implies that there is an uncompleted call in some execution tree (by
lemma \ref{lemma-wr}% and \ref{{lemma-call-chain}}
), and secondly that an uncompleted call
in an execution tree implies that there is an uncompleted call in the
static graph of cycle $C$ used in the detection algorithm (by
\Blue{lemma \ref{lemma-sr}, which states that a node is SR it is in
  all execution trees of cycle $C$.}).
%\Blue{The rest of the proof follows by lemmas ....}

% We first reduce and simplify the set of
%executions without removing flooding of $m$. 
%such that for each flooding execution there is a flooding
%execution in the reduced execution set.
%
\ignore{ Consider an execution flooding wrt.\ a given call of $m$.  We
  will show that there also is an execution that does not stop before
  an enabled step (apart from start nodes) such that it is flooding
  wrt.\ the given call. We consider two cases of enabled steps,
  deterministic (unconditional) statements, and (blocking or
  non-blocking) statements depending on a future (\emph{get}).

   We first observe that for an execution not including such a step,
   the execution extended with the step at some time $t$ when the step
   is enabled, is a possible execution according to the partial order
   semantics (which describes the possible executions).

%1. Deterministic steps
   \begin{description}\item[Deterministic statements]
   (assignments, simple calls, local calls, object generation, and skip):
   \footnote{The partial order semantics describes the possible executions.  It
   captures independent object speeds.  It is obvious that for
   non-interfering, deterministic object execution steps, 
%except when waiting is involved, 
   the final outcomes are the same, regardless of how fast these are executed.
%One may therefore move  non-interfering, deterministic object
%execution steps forward (as long as they are enabled).

    We may pretend that deterministic execution steps take no time
   (not changing the time of  execution of start nodes and continuations 
    of \emph{get}s and \emph{await}s).
}
%of a blocking or suspending wait.
This \emph{simplification} will mean that some calls are produced earlier 
%(which is OK with respect to flooding, i.e., 
%flooding cannot disapear by earlier occurrence of calls), 
and that some \emph{put}s will appear earlier.
This will not cause flooding of $m$ to disappear,
because flooding cannot disappear by earlier
occurrence of calls, and not by earlier \emph{put}s
%The latter will not cause flooding of $m$ to disappear
 because there are
only finitely many objects $y$ that can execute $m$.
Therefore, to find a time where the simplified
execution has  more than $x$ uncompleted $m$ calls,
% in the modified execution, 
we take an original execution at a time where there are $x+y$
uncompleted $m$ calls, % in the origial execution, 
which exists by the assumption of unbounded number of uncompleted $m$ calls.
%
Thus we may assume that at any time $t$ an object $o$ is either idle
or is waiting in a blocking \emph{get}.

%2. gets:  (get or await)
\item[\emph{get} statements in a method other than $m$]
%Consider first method executions other than $m$.
Blocking \emph{get}:
In this case, the  execution of a specific  get is never done,
even after being enabled.
%Doing this step  will not affect $m$ executions if 
By doing the step in a method other than $m$,
$m$ executions  are not affected.
%\item[Suspended \emph{get} statements] 

   Suspended get in a method other than $m$: If done by an object
   being flooded with $m$, $m$ executions may be delayed, for other
   objects $m$ executions are not affected.  In either case, flooding
   is not reduced.  If the object is deadlock...
   \end{description}

   \footnote{(Note that wrt.\ await on condition:
   the algorithm will not know that such conditions are enables,
   and will treat them as unabled.)}
================================= ignore =============================}%
%
\ignore{Thus it suffices to consider executions such that
the \emph{after}-relation is giving priority to active processes as long as
the next step is enabled (i.e., it is not in a blocking get and
unresolved get/await).  Thus a new call is started after all active
processes have reached an idle state or a blocked get (or end of loop,
if the cycle is a loop).
%This is a benefit of the partial order semantics!
Similarly
we may also assume that resumption of suspending \emph{get}s  (await on a
future) are not unnecessarily delayed when the future is available;
thus we do not distinguish blocking and suspended \emph{get}s  wrt.\ delays,
both are assumed to continue without delay when the future value is
present. }
%==========end of reduction of possible executions
%
%\textbf{Note:} In the discussion below we restrict ourselves to simplified executions.
%



Consider a system execution $E$ %, flooding wrt.\ $m$ and
with unbounded iterations of a cycle $C$.
We define the execution tree of iteration i, denoted $T_{E,C,i}$, reflecting 
all executions in $E$  caused by %$C$  during 
 iteration $i$ of the cycle, 
considering all % one or more 
involved objects.

\begin{definition}[Execution Tree]\label{def-tree} 
The execution tree $T_{E,C,i}$  of a system execution $E$ with unbounded  
iterations of the cycle $C$ 
%and with unbounded number of calls to $m$,
is  obtained from  $E$ 
% given by 
by % taking the $i$th iteration of the cycle,
taking the start node of the  $i$th  iteration as the root,
and  including 
\begin{enumerate}
\item \label{def-tree-cycle}
all steps in the cycle iteration except the back edge
\item \label{def-tree-method} \OO{all steps  in $E$
ordered \emph{after} a step already in the tree, according to the 
partial \emph{after} ordering (letting call edges form the branches)}
\item  \label{def-tree-flows} \OO{
the continuation of processes with states already in $T$ over time.}

\ignore{ all states that happen \emph{after} (according to the assumed
  semantic partial order) a node all ready in T and before some other
  state already in T,}
\ignore{ all method executions in $E$ directly or indirectly called
  from the iteration at some time. (These executions form the
  branches).}
\ignore{
  \item \label{def-tree-flows} all steps in these method executions
  as long as  each  \emph{get} or  \emph{await}  is resolved within 
  the tree or by completions before the start of the iteration.}
  % restricted to calls caused by that iteration%
\end{enumerate}\end{definition}
   \ignore{ The execution tree $T_{E,C,i}$ is the set of trees extending 
    the cycle $C$ (minus the back edge) %(without the cycling call) 
    with possible paths through off-cycle parts 
    and with some  invocations, such that
   \begin{enumerate}
   \item each method execution ends at a $put$, or just before an  $await$ or
    a  $get_l$ when the corresponding $put_l$ is not in the tree,
    %at a $get_k$ when $call_k$ is not completed (see below) or at an await,
    and finally
    \item for each $get_l$ in the tree, there is a  $put_l$ in the tree.
   (Unlabeled \emph{get}s  are possible ...)

  \ignore{all states in a process that already has states in T up to the
   but not including a blocked get or await, or through the end of the
   process/method including the put, whichever occurs first.}
   \end{enumerate}}
Clearly  for every start node in
the tree (apart from the root) there is a matching call in the tree.
Flooding calls will in general not be in all  executions  $E$
and therefore not in all execution trees.
%
The tree may be infinite (in case of inner, direct or indirect,
recursion in the cycle) since we do not limit the time. However, we
will be interested in properties of textual occurrences of calls,
and there will only be a finite number of these. %labels involved.
%
Note that according to the object progress assumption, 
  \ignore{all the active method execution of an object ends in an
    unresolved \emph{get} or an \emph{await}.  Enabled processes are
    performed unless there is an unbounded number of them. Thus
    execution trees have a third property
% \begin{enumerate}\setcounter{enumi}{2}
 \item \label{def-tree-flows}}%
\OO{all method executions partly in the cycle must terminate or
 end in an unresolved  \emph{get} or  an \emph{await}.
 It does not ensure that called methods will eventually start to execute.}
\ignore{
   Since there is a finite number of objects involved in the cycle,
   these cannot be blocked. The next cycles would then be blocked.}%
%\end{enumerate}
%
\ignore{
each method execution will end with an unresolved \emph{get} or
\emph{await},
%unless the object has an unbounded number of processes.
\begin{itemize}
\item \label{def-tree-flows} 
all started method executions end in an unresolved
 \emph{get} or  \emph{await}  
     \ignore{ all steps in these method executions as long as each
     \emph{get} or \emph{await} is resolved within the tree or by
       completions before the start of the iteration.}
\end{itemize}}
%objects involved in the cycle....

%Consider all possible system executions. 

%\subsection{ADJUSTMENTS MAY BE NEEDED FROM HERE}
\ignore{%==============================================================
\subsection{Long Version}


\begin{definition}[Object Execution Trace]\label{def-object-ex}
An \emph{execution trace of a given object} of given class at 
a given point in time may be seen as a  finite 
%or infinite 
sequence of execution states, where
%, infinite if the execution does not terminate. 
% Each state is indexed by the corresponding node (label)
% in the program (graph), and
each state includes relevant information such as 
%future identities and 
object state and local state, as well as the program counter value (and label in the program graph), 
and the unique future value of the execution of the current process.
%We say that a state is an s-state if labeled by a statement s.
\ignore{It has a process queue, consisting of (internal and external)
  method executions and remaining parts of processes starting with an
  await on a condition or a future.  A process on the queue is enabled
  if the it starts with an await on a condition that evaluates to
  true, or with an await on a future.  A condition is enabled ...}
The execution reflects method executions performed by the object, and 
must satisfy some basic properties concerning first and next states:
\begin{itemize}
\item
It must start with a state labeled with
the first statement of %a branch through
 the initialization part of the class
(which is the only time this initialization appears in the execution of that object).
\item
The next state after a state with a given program counter value
%The next step after an execution of a statement 
must have the counter value of the next statement
(evaluating if-tests and following the
appropriate branch), 
unless it is a put statement or an await on a condition or a future,
 in which case the next state is the \emph{idle} state,
or a local call, in which case the next state 
has a counter value which 
corresponds to the first statement of the called method;
and in this case the next statement after the end of the 
execution of the locally called method, is the statement following
the local call.

%\item \Blue{After return from a local call, the next statement is the 
%next after the call statement ....}
\item
The next state after idle may only be a start of a  method (of C)
or a state representing the next statement
after a previous await state
of an uncompleted process.
%corresponding to methods performed by the object.
\end{itemize}\end{definition}
Execution of (blocking) \emph{get}s  and idle states 
depends on the  execution of  the whole  system as explained below.

%========================= System Execution  =================================
\begin{definition}[System Execution, System Trace, and Monotonicity]\label{def-system-ex}
An \emph{execution of an object system} (including one initial  object)  is defined by 
\Blue{a set of historically monotone system traces, each reflecting the execution %of each object 
up to  a given time.
As defined below the monotonicity requirement  expresses  that they agree on past traces.
A \emph{system trace} is a partial mapping from object identities to}
object execution traces, one for each object in the system,
together with a set of (shared) futures, given by a partial mapping of
future identities to values (reflecting generated future values), and
with a partial ordering 
(\emph{after}) relating the different object execution traces as
follows: The execution trace of each object $o$ must satisfy
\begin{itemize}
\item the first state of an object of class $C$
(other than the initial object)
 %must %be the beginning of its initiali
must occur \emph{after}
%NOTE during if synchronized ....
the object has been generated by a new $C$ construct,
and must be the start of its initialization part (with an object state
reflecting class parameters)
\item the start of a method execution must
appear \emph{after} the corresponding call, and the call must be on $o$
\ignore{\item the next step of an execution of $o$ must  be the next statement
in the branch of the active process (evaluating if tests and following the
appropriate branch), unless it is an await or get statement, or local call.}
\item a blocking get must be executed \emph{after} the future value has been generated
\item 
a state representing the next statement \emph{after} an await of a process $p$
must occur \emph{after} an idle state when the await is enabled (and when it is the next 
state of the process $p$). % and when the previous state reaches an idle state.
\ignore{
if the next statement is an await the remaining process (including
  the await and and the local state) is placed on the process queue of the
  object, allowing the object to proceed with another process on the
  queue, provided it is enabled.}
\item when a put is executed (the current process is said to be
  finished and) the future  mapping is extended
by associating the produced
value with the   future identity (and we say that the future value has been
  generated). The object is in \emph{idle} state.
\item The execution trace of objects not generated (apart from the initial object)  is $\bot$,
and the future value of futures not (yet) produced is $\bot$.
\end{itemize}
The set of system states must be historically monotonic
in the sense that any two  system states $s1$ and $s2$ must satisfy $s1\leq s2$ or  $s2\leq s1$,
where $s1\leq s2$ means that 
$s1[oid] \not=\bot  \Rightarrow s1[oid] \leq s2[oid]$, for each object identity $oid$
and $s1[\mathit{fid}] \not=\bot  \Rightarrow s1[\mathit{fid}] =s2[\mathit{fid}]$, for each future identity $\mathit{fid}$.
\end{definition}
%
\ignore{OLD: 
An infinite execution is captured by an unbounded set of finite execution sets
(one for each object)
such that the set is prefix-closed (i.e.,  the set  of all finite initial parts of the execution).
We say that an execution is flooding if
An infinite execution is captured by an unbounded set of finite executions
such that the set is prefix-closed (i.e.,  the set  of all finite initial parts of the execution).}

\begin{definition}[Unbounded Number of Calls]\label{def-unbounded}
A  system execution is said to produce \emph{an unbounded number of calls} (of a certain kind)
if for every bound it has a trace with a higher number of  calls (of the given kind).
\end{definition}

\begin{definition}[Flooding]\label{def-flooding-ex}
An execution is flooding if it produces an unbounded number of
uncompleted method calls.
It is flooding wrt.\ a method $m$ if it produces an unbounded number of
uncompleted method calls to  $m$.
\end{definition}


The theorem below expresses that our algorithm covers all possible floodings
(no false negatives).

\begin{theorem}[No false negatives]
\label{theorem-main}
If there is an execution of a given program which is flooding wrt.\ a method $m$, 
our algorithm detects flooding of a  call to $m$ or of 
a call  $m'$ such that flooding of $m'$ implies flooding of $m$.
\end{theorem}

\textbf{Notation}
of nodes: we write $call_n, get_n, put_n$, and $start_n$
where $n$ is the label of the corresponding call. 
%Similarly start nodes are indexed by the label of the corresponding call.


%\textbf{Proof: }
\begin{proof}
Assume that there is an execution $E$ of a given program with flooding
wrt.\ a method $m$.  We need to prove that flooding of a call to $m$
is detected (or at least a call $m'$ such that flooding of $m'$
implies flooding of $m$).


Consider an execution $E$ which is flooding  wrt.\ to method $m$.  Since
there are finitely many $m$ calls in the code, the execution must also
be flooding  wrt.\ one of these calls, say labeled $n$.  Thus it suffices
to consider each textual call rather than each method.  Let us assume
that $call_n$ is flooding in given execution $E$.
%%For  $E$  the call $call_n$
This call must be caused by a cycle $C$, repeated without bound.  (%In $E$ 
The cycle may be interleaved with other flooding or non-flooding
cycles.)  Thus the call $call_n$ must occur in a method
which is called (directly or indirectly) from an iteration of  the cycle,
i.e., \emph{there must be a call path from the cycle to the call}.
%the call  $n$  may be off-cycle).
%
\textbf{Note:}
Thus flooding wrt.\ $m$ in the sense of \ref{def-flooding-ex}  implies 
flooding wrt.\ a cycle in the sense of \ref{flooding-cycle}\,. 


Consider a repeated cycle involving one or more processes.
At the time when the next iteration of the cycle starts there may be some remaining parts
of processes started in the previous  iteration of the  cycle. 
%Since these are processing have the any remaining bort
Flooding occurs when an object over time gets more processes to do  
than it can handle. Each iteration of a cycle may
generate new calls on some objects and may leave some unfinished
processes. However, the unfinished processes are already active,
and new processes on that object  generated in  the next cycle cannot start
until the previous ones are finished or suspended.
Assuming there is a bounded number of objects, the same objects will
be involved in  future cycles, and if one of these %that object 
is busy, the cycling will need to wait.
\textbf{Note:} Thus flooding due to blocking \emph{get}s  is not a problem for objects
involved in the cycle.\footnote{
   NOT NEEDED?  Thus flooding can be caused by called methods
   being delayed, or by suspended parts of remaining processes
   being disabled.}%
\footnote{DO WE NEED THIS? 
   Thus it suffices to look at calls generated by a cycle and remaining
   parts of processes \emph{after} suspension.  Remaining processes without
   suspension do not cause flooding.  And suspension while waiting for
   a future can not cause flooding if the future value is available
   (assuming that a suspending get has priority over new calls when
   enabled).  Thus flooding may be caused by calls generated by the
   cycle, remaining parts of processes after an await or a get on a
   future not ready.  }%
\footnote{MAY NOT NEED ANY LONGER? We may need to assume there is no
  deadlock.  With respect to possible executions: we assume that a
  method execution that has started continues until the end or to an
  internal loop or to an await with a false condition or to a get
  where the future value is not available.  A called method may be
  delayed (reflecting that process speeds are independent and may also
  depend on scheduling).}
%
\ignore{REUSE DEFINITION IN NEW CONTEXT, SKIP OTHER PARTS: 
 The cycle $C$ can be seen as a sequence of nodes starting with a start
 node of some method $p$ (or start node of a loop) following one
 possible path through that method (loop). At a call it may continue
 with the start node of the called method (following one possible path
 through that method) or continue with the path through $p$, and so on
 (involving zero or more methods calls), until it reaches a call back
 to $p$ (or end of the loop body).  There may be some remaining parts
 of the methods started during the cycle, called \emph{off-cycle part}
 (e.g.\ $put:2$ in fig.\ \ref{graph-modified}), and there may be some
 method calls leading out of the cycle (e.g.\ $Pd:1$ in
 fig.\ \ref{graph-modified}) . % ("off-cycle calls").
 The flooding call $call_n$ must be in the cycle, in off-cycle parts, or 
 in methods directly or indirectly called from the cycle or off-cycle parts. 

 Consider the set of execution trees $T_{C,i}$ reflecting possible
 executions during one iteration $i$ of the cycle, considering one or
 more involved objects.  Execution of called methods are included if
 they are started during the cycle (before the next iteration) and
 suspension is ignored if resolved during the cycle iteration.
 \begin{definition}[Execution tree]\label{def-tree}
  The set of execution trees $T_{C,i}$ is the set of trees extending
  the cycle $C$ (minus the back edge) with possible paths through
  off-cycle parts and with some  invocations such that
 \begin{enumerate}
  \item $call_n$ (the call causing the flooding) is in the tree, and 
  \item for every start node there is a matching call in the tree,    and  
 %(apart from the initial start node of $p$) %every invocation is called from the  tree,
  \item each method execution ends at a $put$, or just before an
   $await$ or a $get_l$ when the corresponding $put_l$ is not in the
   tree,
 %at a $get_k$ when $call_k$ is not completed (see below) or at an await,
  and finally
  \item for each $get_l$ in the tree, there is a  $put_l$ in the tree.
 \end{enumerate}\end{definition}
 The first point says that we are only considering flooding iterations;
 the second that only relevant calls are considered; the third that an
 unfinished execution must be in an unresolved waiting condition; and
 the last point says that the tree is self-contained in the sense that
 it includes all method executions known to complete, and all nodes of
 the chosen path of that execution must be in the tree, including
 $get$s to other methods (which may in turn mean that yet other
 \emph{put}s must be in the tree).  A call $call_l$ without a start
 node in the tree (in which case there is no $get_l$ in the tree),
 reflects that the call was delayed until after the cycle.
 = END IGNORE =====================}

%\textbf{Note:} 
The partial order semantics describes the possible
executions.  It captures independent object speed.  If an execution is
flooding with respect to a method $m$ for some execution $E$, it is
also flooding in executions where the starts of executions of $m$ are
 (more or less) delayed.  
\Blue{It is obvious that for non-interfering, deterministic object
execution steps, the final outcomes are the same, regardless of how
these are ordered.}  Thus it suffices to consider executions such that
the \emph{after}-relation is giving priority to active processes as long as
the next step is enabled (i.e., it is not in a blocking get and
unresolved get/await).  Thus a new call is started after all active
processes have reached an idle state or a blocked get (or end of loop,
if the cycle is a loop).
%\footnote{this is a benefit of the partial order semantics!}
Similarly
we may also assume that resumption of suspending \emph{get}s  (await on a
future) are not unnecessarily delayed when the future is available;
thus we do not distinguish blocking and suspended \emph{get}s  wrt.\ delays,
both are assumed to continue without delay when the future value is
present. \textbf{Note:} In the discussion below we restrict ourselves to such 
executions.



Consider a system execution  $E$ with unbounded iterations of a cycle $C$.
We define the execution tree of iteration i, denoted $T_{E,C,i}$, reflecting 
all executions in E caused by C during  iteration $i$ of the cycle, 
considering one or more involved objects.
\footnote{OLD TEXT  - REDO:
 Execution of called methods (identified by the future identities  of the calls) are included if they are started 
 in $E$
% during   the cycle (before the next iteration) 
and suspension is ignored if resolved during the cycle iteration.}
 
\begin{definition}[Execution Tree]\label{def-tree} 
(MAY NEED TO ADJUST A BIT MORE!)
The execution tree $T_{E,C,i}$ 
 of a given system trace $E$
 with at least $i$ iterations of the cycle and with at least
 $i$  calls to the  flooding method,
 is the system trace $E$ 
 given by the $i$th iteration of the cycle
 (without the start of the next iteration), 
 restricted to calls caused by that iteration%
 \ignore{  The execution tree $T_{E,C,i}$ is the set of trees extending 
  the cycle $C$ (minus the back edge) %(without the cycling call) 
  with possible paths through off-cycle parts 
  and with some  invocations},
such that
\begin{enumerate}
% \item $call_n$ (the call causing the flooding) is in the tree, and 
 \item for every start node there is a matching call in the tree,  %(apart from   the initial start node of $p$)
  %every invocation is called from the tree,
  and  
 \item each method execution ends at a $put$, or just before an  $await$ or
  a  $get_l$ when the corresponding $put_l$ is not in the tree,
  %at a $get_k$ when $call_k$ is not completed (see below) or at an await,
  and finally
  \item for each $get_l$ in the tree, there is a  $put_l$ in the tree.
(Unlabeled \emph{get}s  are possible ...)
\end{enumerate}

\end{definition}
Note that 
$call_n$ (the call causing the flooding) is in the tree,
(since %The first point says that 
we are only considering  flooding iterations).
the first point says  that only relevant calls are considered;
the second that an unfinished execution must
be in an unresolved waiting condition; and 
the last point says that 
the tree is self-contained in the sense that it includes
%we must include 
all method executions known to complete,
and all nodes of the chosen path of that execution must be in the tree,
including $get$s to other methods (which may in turn mean that yet other
\emph{put}s must be in the tree). 
A call $call_l$  without a start node in the tree
 (in which case there is  no $get_l$ in the tree), reflects that 
the call was delayed until after the cycle.
\footnote{since in the set of traces we may stop before a start node.}



A tree will be finite since there are finitely many labeled calls,
and each labeled call can only give rise to one 
start node and corresponding  path through the method.

}%=====================end ignore LONG VERSION ==========================

\begin{definition}[Execution Tree Set]\label{def-tree-set}
 The set of execution trees
 $T_{C}$ 
is the set of execution trees $T_{E,C,i}$ 
for all possible system executions $E$ 
with an unbounded number of $C$ iterations, and for any possible iteration $i$.
%traces E of a system execution  with an unbounded number of C iterations,
%and for any possible iteration i.
\end{definition}
%
\ignore{MERGE WITH EARLIER TEXT??
The set of possible trees reflect
that continued execution of the remaining part of a  method 
is fast as long as there is no suspension (i.e. no time).
 Suspension may take no time or non-trivial time. Similarly
execution of a new method may take no time or non-trivial time.
Thus an execution tree may or may not be cut at any such  point.
We assume here that a get on a completed call takes "no time".}
%may or may not take Since time needed in cases of suspension or
%invocation depends on the execution and may range from zero time to
%non-trivial time.
\ignore{ The set of trees $T_{C}$ represents possible executions of an
  iteration of cycle $C$ where method executions not in the tree are
  delayed.  For each iteration of of the cycle in an execution, there
  will be a tree, reflecting the activity related to the iteration up
  to the point where the next iteration starts.
     A call with label $k$ is said to be \emph{completed} in a tree
  $T_{E,C,i}$ if a $put_k$ or $get_k$ is in the tree.
}% end ignore
%
A call with label $k$ is said to be  (possibly) reachable if  
$call_k \in %%%%%%%%%%%%%%%%\bigcup_{E,i}  
T_{E,C,i}$ for some $E$ and $i$.
A call $k$ is (surely) completed if 
$put_k \in %%%%%%%%%%%%%%%\bigcap_{E,i}  
T_{E,C,i} $ for all  $E$ and $i$.\footnote{Or $get_k$?
No, there must be a put before get}
% or  $get_k\in \bigcap_i T_{C,i} $
We define $\callsE$ by the label set $\{n \,|\, \exists\ E,i \ .\  call_n\in T_{E,C,i}\}$, and 
$\compsE$ by $\{n \,|\,  \forall\, E,i\ .\  put_n\in T_{E,C,i}\}$.

We next show that  $ \callsE$ is 
%the same as the 
 %overapproximates 
contained in the set of  calls   detected,  
and  that all detected  $comps$ are in 
$\compsE$. % is the same as  the detected $comps$.
%(TODO: this is  not quite true since we may detect a call m' such that
%flooding of m' implies flooding of $call_n$).
Thus if 
a call $n$  is flooding in some execution,
$n$ is in  $ \callsE-\compsE$, and thus
in  $ \callsD-\compsD$, i.e., the flooding call  must be detected.




\ignore{====================OLD=============
    In the analysis below we look at reachable nodes in the graph
    $G_{C,E}$ consisting of the cycle $C$ (following only the chosen
    path where there is branching), remaining method parts (including
    all paths), and also called methods.  The nodes of $G_{C,E}$ are
    reachable from the cycle and the set of calls in $G_{C,E}$ is
    called the reachable calls of the cycle, denoted $ calls_{C}$.

   $call_n\in G_{C,E}$

   $\callsE= \{n| call_n \in G_{C,E}\}$

   For given execution E and cycle C, we may look at all calls
   reachable from an iteration of the cycle, say $calls_{E,C}$. We
   must have $n\in calls_{E,C}$ And if we look at all completions,
   $comps_{E,C}$, that are made in every iteration of the cycle, we
   must have $n\not\in comps_{E,C}$.  I.e., the call is in the cycle
   or reachable from the cycle, and the call is not completed an
   unbounded number of times.


   We next define $ calls_{C}$ and $comps_{C}$ such that the former
   over-approximates $ calls_{E,C}$ and the latter under-approximates
   $comps_{E,C}$.  Thus if $n$ is in $ calls_{E,C}-comps_{E,C}$ it
   must also be in $ calls_{C}-comps_{C}$.

   $ calls_{C}$ consists of all calls $n$ such that

  \begin{itemize}
  \item call $n$ is in $C$
  \item  call  $n$  is in some  off-cycle path of $C$
  \item call $n$ is in some method called by $C$ (directly or
  indirectly)
  \end{itemize}
   $ comps_{C}$ consists of all calls $n$ such that
  \begin{itemize}
  \item put  $n$  is in $C$
  \item get  $n$  is in $C$
  \item put/get $n$ is reachable in each path of an off-cycle path of $C$
  \item put/get $n$ is  in each path of a method in $ calls_{C}\cap  comps_{C}$
  \end{itemize} %if a call i is in the cycle it is in
}
\ignore{Note:
 consider the cyclic program
 m(){if ... then f:= call m1 else f:= call m2 fi;
    get f; call m (): put}
 Here we would need two put-get edges, one from m1 and one from m2
}
\footnote{\textbf{Note:} There could be a node $get\, f$ in the cycle
  where the future f is defined outside the cycle, say by a parameter
  or a field.  Will this be a Problem for our definitions?  \Blue{Such
    a $get\ f$ can be handled as an $await$ since completion cannot be
    detected.}  }
%{\textbf{Note:}}
\ignore{
\subsection{MAY NOT NEED:  Alternative  Executions}
\emph{Motivation:  Rather than trees $ T_C,j$
  we  consider initial parts of executions
  (which is the real thing when it comes to semantics).}

   An \emph{execution} can then be seen as a finite (if the execution
   stops) or infinite (if the execution does not stop) sequence of
   states, augmented with relevant information about executing nodes
   etc.  Flooding of a cycle $C$ wrt.\ a method $m$ would mean that
   there are one or more infinite executions where $C$ is repeated
   infinitely and there are infinite number of calls to $m$.

   We assume that each process has no infinite sequence of execution
   steps until the process reaches a \emph{resting} state,
   i.e.\ process end, a blocking get, or a suspension point (await).

   %If the call back to the cycle is a local, blocking call 
   \ignore{ For such an execution $E_C$ we let $E_C,n$ mean the
     finite, initial part of $E_C$ ending just before the $n+1$th
     iteration of $C$, letting each object (apart from the one doing
     the start of the cycle -- in case of recursion?) reach a resting
     state.  (If there is flooding of $C$ wrt.\ a method $m$, there is
     also flooding of $C$ wrt.\ a method $m$ for an execution of this
     kind, since all possible object speeds must be considered.)}  }

\ignore{We may continue more or less as before.  For each
  iteration/occurrence of the cycle $C$ in such an execution $E$, we
  define the associated execution tree $T_{E,C}$ which contains the
  occurrence of the cycle $C$ extended with call edges and start nodes
  for called methods starting to execute before the next iteration of
  the cycle, letting each execution end in a waiting, blocking or
  ending node.  By induction on the set of execution we know that ...
  same def as before (more or less)...  to be completed.}


%\paragraph{Observation:}
\begin{lemma}
\label{lemma-await}
 We may ignore executions where await on a future is 
treated different than get, in the sense that the set of floodings
in the first case is not larger than the set of floodings in the 
latter case.
\end{lemma} 
\begin{proof}
We first observe that 
with the under-specified scheduling of our partial order semantics,
a possible way of implementing suspending gets is
by implementing them as a (blocking) get.
It then suffices to prove that 
executions different from those using a blocking get,
do not provide more flooding.

A method execution of blocking rather than suspending
may contribute to flooding since it may not terminate.
However, if there are unbounded many calls to such methods,
the method executions can be delayed before start, rather than
at an await. 
In case the there is a call back in the first part of the method
(before a suspending get), 
this may or may not be the call back of the current cycle $C$.
In the first case, the executing object 
the completion of the current call is not strongly reachable,
so flooding in this case is detected.
In the second case,
we have a shorter cycle that is flooding.
Therefore  flooding of this method is already possible.

A call $n$ occurring after a suspending get, may never be
performed with executions that perform  suspending gets by blocking.
% rather than suspend. 
 However, these calls will be weakly reachable
and thus count in the static  detection of flooding.
The same argumentation applies to calls indirectly caused by %the 
call $n$.
%path starting after a suspended get.


Anything else...

\ignore{
Consider an execution $E$ where the execution of 
an await on a future in a given process %method instance 
is reached at time $t$.
It could be that the await is never enabled,
or that it is  enabled at time $t'>t$
(in which case the process will continue to be enabled until executed).
In the latter case 
the await may 
never be executed or it may be executed at time  $t''>t'$.
Compare this execution to an execution $E'$ where the await statement
is executed like a blocking get.
%, at some time $t`>t$, and such that the two exeutions are
%equivalent up to time $t'$.

If the process represent the flooding method ...}
\end{proof}



\begin{lemma}
\label{lemma-sr}
If node $N_i$ is \emph{strongly-reachable} (SR) in graph $G$ with respect to cycle $C$ then $N_i$ is in every $T_{E,C,j}$ as defined in 
definition \ref{def-tree}\,.
\end{lemma}

\begin{proof}
We will use structural induction on the set of SR nodes to prove that Lemma \ref{lemma-sr} 
holds for all SR nodes as defined by definition \ref{defn-sreachable}\,.
\\ \emph%\subsubsection
{Base case:}
Let the set $S$ of SR nodes contain just nodes that are in $C$ (defn: \ref{defn-sreachable}.\ref{sr-incycle}).
Clearly all nodes in $S$ will be in $T_{E,C,j}$ by definition of $T_{E,C,m}$ (definition \ref{def-tree}\,).\footnote{What is $m$ in  $T_{E,C,m}$}
%
\\ \emph%subsubsection
{Induction step:}
Assume Lemma \ref{lemma-sr}  holds for some set $S$ of SR nodes. We will show that applying each of the subparts of defn: \ref{defn-sreachable}
defines a node to be SR for which the lemma holds.

\ref{defn-sreachable}.2: $N_j$ is not a blocking node. $N_i$ is in every $T_{E,C,k}$ by the inductive assumption. 
The only way to continue execution of the method containing $N_i$ is to execute $N_j$. Therefore 
$N_j$ must also be in $T_{E,C,k}$ (definition \ref{def-tree}.c).

\ref{defn-sreachable}.3: The argument is the same as for the previous case with the addition of the assumption that the \emph{put}
indicating the completion of any possible future needed by this \emph{get}
must also be SR (and in the tree). Since the only way to continue the execution of the method containing $N_i$
is $N_j$ and $N_j$ is unblocked due to the completion of its future, $N_j$ must be in $T_{E,C,i}$ (definition \ref{def-tree}.c).

\ref{defn-sreachable}.4: From the assumption of $n \in {\compsD}$, the method corresponding to
${start}_n$ has finished and it cannot finish without starting, thus
the node ${start}_n$ must have executed before the end of the tree. 
Also by assumption, the corresponding node ${call}_n$ is SR and in $T_{E,C,m}$, thus the node ${start}_n$ must also
be in $T_{E,C,m}$, satisfying the lemma.

\ref{defn-sreachable}.5: The nodes ${start}_n$ and $N_k$ (a ${get}_n$ node) are SR and in $T_{E,C,m}$ (by assumption). But ${get}_n$ can only execute
if ${put}_n$ has executed, thus the ${put}_n$ node $N_j$ must also be in $T_{E,C,m}$.

\ref{defn-sreachable}.6: The only way to get to node $N_k$ is through $N_j$ and by assumption $N_k$ is SR and in every $T_{E,C,m}$.
Thus the only way for $N_j$ to not be in $T_{E,C,m}$ would be for $N_k$ to be the first node in its method in $T_{E,C,m}$.
But this is impossible.
$N_k$ cannot be a \emph{start} node as \emph{start} nodes are not flow-reachable from any node. 
If $N_k$ is the first node in its method to execute in $T_{E,C,m}$ (i.e. although we must go through $N_j$ to get to $N_k$ maybe
$N_j$ is not in $T_{E,C,m}$), $N_k$ must be in the cycle which implies some other node $N_i$ reaches $N_k$ (to close the cycle) 
which contradicts the assumption that $N_k$ is not flow-reachable from any node other than $N_j$.
\end{proof}

\begin{lemma}
\label{lemma-call-chain}
If node $call_n$ is \emph{flooding} with respect to cycle $C$ in graph $G$ then there must be a call chain from a \emph{call}
in $C$ to $call_n$.\footnote{
Moreover, the start node of the  method called by e $call_n$
 is weakly reachable.}
\end{lemma}

\begin{proof}%Proof: 
This %The first part of this 
lemma follows directly from definition  \ref{flooding-cycle}\,.
\footnote{The second part follows by simple inductin on the length of the call chain
using  definition \ref{defn-wreachable}.\ref{wr-calledge}\,.}
\end{proof}

\begin{lemma}
\label{lemma-no-flood-chain}
If $\exists$ a call chain ${call}_0, {call}_1, ... {call}_n$ such that ${call}_0 \in C$ and all ${call}_i, i <= n$ are not flooding,
then all ${start}_i, i <= n$ are SR.
\end{lemma}

\begin{proof}%Proof: 
We will use structural induction on the length of the call chain.\\
\emph{Base case:} If the length of the call chain is zero, then $call_0$ is SR by definition \ref{defn-sreachable}.\ref{sr-incycle}\,.
Because $call_0$ is not flooding, $n$ must be in $\compsD$. Therefore
${start}_0$ is SR by definition \ref{defn-sreachable}.\ref{sr-start}\,.
\\
\emph{
Induction step:} If the length of the call chain, $n$, is greater than zero, then assume all ${start}_i, i < n$ are SR.
However, the only way for ${start}_i$ to be SR is for ${call}_i$ to be SR (definition \ref{defn-sreachable}.\ref{sr-start}), therefore
all ${call}_i, i < n$ are SR. Because ${start}_{n-1}$ is SR then by definition \ref{defn-wreachable}.\ref{wr-start} ${call}_n$ is weakly
reachable and thus $n \in {\callsD}$, and from the assumption of the lemma, ${call}_n$ is not flooding,
it must be the case that $n \in {\compsD}$. 
Therefore, by definition \ref{defn-sreachable}.\ref{sr-start}, 
${start}_n$ is SR.
\end{proof}

\begin{lemma}
\label{lemma-wr}
If node $call_n$ is \emph{flooding} with respect to cycle $C$ in graph $G$ then $n \in {\callsD}$\ignore{THE FOLLOWING PART CAN NOW BE REMOVED, RIGHT? or
$\exists call_i$ that is flooding, there is a call chain from $call_i$ to $call_n$, and $i \in {\callsD}$}.
\end{lemma}

\begin{proof}%Proof: 
We will use  induction on the length of the call chain from the call $call_0 \in C$ to $call_n$, where the
call chain is $call_0, call_1, ... call_n$.
\\
\emph{Base case}: 
If the length of the call chain is zero, then $call_0$ is the flooding
call which by definition \ref{defn-sreachable}.\ref{sr-incycle} is SR
and thus by definitions \ref{defn-wreachable} and \ref{defn-calls}, $0
\in {\callsD}$.
\\
\emph{Induction step}: 
If the length of the call chain is $n+1$ (with $n\geq 0$),
then we may assume $n \in {\callsD}$.
By definitions  {\ref{defn-wreachable}.\ref{wr-calledge}}
and {\ref{defn-wreachable}.\ref{wr-start}}
we have that ${n+1} \in {\callsD}$.
\ignore{OLD TEXT WITH OLD DEF OF WR:
  If the length of the call chain, $n$,
  is greater than zero,  then assume 
  $i \in {\callsD}$ for all $call_i$ in the chain $i < n$.
  %It there is an $i<n$ such that $call_i$ is flooding the lemma holds.
  %Otherwise, 
  \Blue{We must then prove $n\in \callsD$.  There are two cases to
   consider.  First, if there is an $i<n$ such that $call_i$ is
   flooding then by the inductive hypothesis $i \in {\callsD}$ and there
   is a call chain from $call_i$ to $call_n$, satisfying the lemma.}
  In the other case, $call_n$ is the first call in the chain to be
  flooding.  By assumption the method with $start_{n-1}$ contains the
  call $call_n$.  $\forall i < n, {call_i}$, is not flooding thus
  $call_i$ is SR. \Blue{(I forget why not flooding implies SR. I guess
   we know it must be called and since it isn't flooding, it must
   finish. If it finishes its put is SR and that gets back to it being
   SR.  Do I need yet another lemma here?)}  and $\forall i < n,
 {call_i}$ is $SR$.

%(do I need to elaborate here? \Blue{YES, HOW ABOUT:
%  None of the  calls $call_i$ are flooding for $i<n$ and thus
%all start nodes  $start_i$  for $i<n$ are in SR}).

  By definition \ref{defn-wreachable} every node in the method
  containing $start_{n-1}$ is \emph{weakly-reachable} which includes
  the node $call_n$. Thus $call_n$ is \emph{weakly-reachable} and by
  definition \ref{defn-calls} is in ${\callsD}$.
}
\end{proof}

\subsection{Proof of Theorem \ref{thm-flooding}}
\begin{proof}
If there is some \emph{flooding} call, $call_n$, where $n \not\in {\callsD}-{\compsD}$ then either
\begin{description}
\item[a)] $n \not\in {\callsD}$, or
\item[b)] $n \in {\compsD}$.
\end{description}
By Lemma \ref{lemma-call-chain}, there must be a call chain starting on the 
cycle that leads to $call_n$ and by Lemma \ref{lemma-wr} $n \in {\callsD}$. 
%(Note: Need to correct of this being the first in a chain of flooded calls.)  
Therefore if the theorem does not hold, it must be because $n \in {\compsD}$ 
(and shouldn't be).

If $n \in {\compsD}$ then from definition \ref{defn-comps} and lemma
\ref{lemma-sr}, we know that either the method with label $n$ has
finished ($put_n$ or $get_n$ are SR), or the method with label $n$ is
partially executed directly in the cycle and none of the flow-paths
leading from the last node of the method in the cycle will suspend
before reaching a put.  Therefore $call_n$ cannot be flooding.
\end{proof}
%========================================

\subsection{Proof of Theorem \ref{theorem-main}}
\begin{proof}
%\noindent\textbf{Proof of theorem \ref{theorem-main}.}
Assume that there is a system execution $E$ of a given program with
flooding wrt.\ a method $m$.  We need to prove that flooding of a call
to $m$ is detected (or at least a call $m'$ such that flooding of $m'$
implies flooding of $m$).

Consider a system execution $E$ which is flooding  wrt.\ to method $m$.  Since
there are finitely many $m$ calls in the code, the execution must also
be flooding  wrt.\ one of these calls, say labeled $n$.  Thus it suffices
to consider each textual call rather than each method.  Let us assume
that $call_n$ is flooding in given execution $E$.
%%For  $E$  the call $call_n$
This call must be caused by a cycle $C$, repeated without bound.  (%In $E$ 
The cycle may be interleaved with other flooding or non-flooding
cycles.)  Thus the call $call_n$ must occur 
in the iteration of the cycle, or in 
a method called (directly or indirectly) from the iteration,
i.e., there must be a call path from the cycle to the call.
%the call  $n$  may be off-cycle).
Thus flooding wrt.\ $m$ in the sense of \ref{def-flooding-ex}  
implies flooding wrt.\ a cycle in the sense of \ref{flooding-cycle}\,.
The theorem therefore follows by theorem \ref{thm-flooding}\,.
%proved below. 
\end{proof}
%=======================================================================

\ignore{%=============================================================== 
  Before proving \ref{thm-flooding} we first reduce and simplify the
  set of executions without removing flooding of $m$.
  \begin{lemma}[Reduced execution set]
  \label{lemma-reduced-set}
  If there is an execution of a given program which is flooding wrt.\ a
  method $m$, our algorithm detects flooding of a call to $m$ or of a
  call $m'$ such that flooding of $m'$ implies flooding of $m$.
  \end{lemma}
  \noindent\textbf{Proof of lemma \ref{lemma-reduced-set}.}
  }%====================================================================


% LocalWords:  wrt
 %separate file
%\subsection*{Further lemmas}
%\begin{lemma}
\label{lemma-sr}
If node $N_i$ is \emph{strongly-reachable} (SR) in graph $G$ with respect to cycle $C$ then $N_i$ is in every $T_{E,C,j}$ as defined in 
definition \ref{def-tree}\,.
\end{lemma}

\begin{proof}
We will use structural induction on the set of SR nodes to prove that Lemma \ref{lemma-sr} 
holds for all SR nodes as defined by definition \ref{defn-sreachable}\,.
\\ \emph%\subsubsection
{Base case:}
Let the set $S$ of SR nodes contain just nodes that are in $C$ (defn: \ref{defn-sreachable}.\ref{sr-incycle}).
Clearly all nodes in $S$ will be in $T_{E,C,j}$ by definition of $T_{E,C,m}$ (definition \ref{def-tree}\,).\footnote{What is $m$ in  $T_{E,C,m}$}
%
\\ \emph%subsubsection
{Induction step:}
Assume Lemma \ref{lemma-sr}  holds for some set $S$ of SR nodes. We will show that applying each of the subparts of defn: \ref{defn-sreachable}
defines a node to be SR for which the lemma holds.

\ref{defn-sreachable}.2: $N_j$ is not a blocking node. $N_i$ is in every $T_{E,C,k}$ by the inductive assumption. 
The only way to continue execution of the method containing $N_i$ is to execute $N_j$. Therefore 
$N_j$ must also be in $T_{E,C,k}$ (definition \ref{def-tree}.c).

\ref{defn-sreachable}.3: The argument is the same as for the previous case with the addition of the assumption that the \emph{put}
indicating the completion of any possible future needed by this \emph{get}
must also be SR (and in the tree). Since the only way to continue the execution of the method containing $N_i$
is $N_j$ and $N_j$ is unblocked due to the completion of its future, $N_j$ must be in $T_{E,C,i}$ (definition \ref{def-tree}.c).

\ref{defn-sreachable}.4: From the assumption of $n \in {\compsD}$, the method corresponding to
${start}_n$ has finished and it cannot finish without starting, thus
the node ${start}_n$ must have executed before the end of the tree. 
Also by assumption, the corresponding node ${call}_n$ is SR and in $T_{E,C,m}$, thus the node ${start}_n$ must also
be in $T_{E,C,m}$, satisfying the lemma.

\ref{defn-sreachable}.5: The nodes ${start}_n$ and $N_k$ (a ${get}_n$ node) are SR and in $T_{E,C,m}$ (by assumption). But ${get}_n$ can only execute
if ${put}_n$ has executed, thus the ${put}_n$ node $N_j$ must also be in $T_{E,C,m}$.

\ref{defn-sreachable}.6: The only way to get to node $N_k$ is through $N_j$ and by assumption $N_k$ is SR and in every $T_{E,C,m}$.
Thus the only way for $N_j$ to not be in $T_{E,C,m}$ would be for $N_k$ to be the first node in its method in $T_{E,C,m}$.
But this is impossible.
$N_k$ cannot be a \emph{start} node as \emph{start} nodes are not flow-reachable from any node. 
If $N_k$ is the first node in its method to execute in $T_{E,C,m}$ (i.e. although we must go through $N_j$ to get to $N_k$ maybe
$N_j$ is not in $T_{E,C,m}$), $N_k$ must be in the cycle which implies some other node $N_i$ reaches $N_k$ (to close the cycle) 
which contradicts the assumption that $N_k$ is not flow-reachable from any node other than $N_j$.
\end{proof}

\begin{lemma}
\label{lemma-call-chain}
If node $call_n$ is \emph{flooding} with respect to cycle $C$ in graph $G$ then there must be a call chain from a \emph{call}
in $C$ to $call_n$.\footnote{
Moreover, the start node of the  method called by e $call_n$
 is weakly reachable.}
\end{lemma}

\begin{proof}%Proof: 
This %The first part of this 
lemma follows directly from definition  \ref{flooding-cycle}\,.
\footnote{The second part follows by simple inductin on the length of the call chain
using  definition \ref{defn-wreachable}.\ref{wr-calledge}\,.}
\end{proof}

\begin{lemma}
\label{lemma-no-flood-chain}
If $\exists$ a call chain ${call}_0, {call}_1, ... {call}_n$ such that ${call}_0 \in C$ and all ${call}_i, i <= n$ are not flooding,
then all ${start}_i, i <= n$ are SR.
\end{lemma}

\begin{proof}%Proof: 
We will use structural induction on the length of the call chain.\\
\emph{Base case:} If the length of the call chain is zero, then $call_0$ is SR by definition \ref{defn-sreachable}.\ref{sr-incycle}\,.
Because $call_0$ is not flooding, $n$ must be in $\compsD$. Therefore
${start}_0$ is SR by definition \ref{defn-sreachable}.\ref{sr-start}\,.
\\
\emph{
Induction step:} If the length of the call chain, $n$, is greater than zero, then assume all ${start}_i, i < n$ are SR.
However, the only way for ${start}_i$ to be SR is for ${call}_i$ to be SR (definition \ref{defn-sreachable}.\ref{sr-start}), therefore
all ${call}_i, i < n$ are SR. Because ${start}_{n-1}$ is SR then by definition \ref{defn-wreachable}.\ref{wr-start} ${call}_n$ is weakly
reachable and thus $n \in {\callsD}$, and from the assumption of the lemma, ${call}_n$ is not flooding,
it must be the case that $n \in {\compsD}$. 
Therefore, by definition \ref{defn-sreachable}.\ref{sr-start}, 
${start}_n$ is SR.
\end{proof}

\begin{lemma}
\label{lemma-wr}
If node $call_n$ is \emph{flooding} with respect to cycle $C$ in graph $G$ then $n \in {\callsD}$\ignore{THE FOLLOWING PART CAN NOW BE REMOVED, RIGHT? or
$\exists call_i$ that is flooding, there is a call chain from $call_i$ to $call_n$, and $i \in {\callsD}$}.
\end{lemma}

\begin{proof}%Proof: 
We will use  induction on the length of the call chain from the call $call_0 \in C$ to $call_n$, where the
call chain is $call_0, call_1, ... call_n$.
\\
\emph{Base case}: 
If the length of the call chain is zero, then $call_0$ is the flooding
call which by definition \ref{defn-sreachable}.\ref{sr-incycle} is SR
and thus by definitions \ref{defn-wreachable} and \ref{defn-calls}, $0
\in {\callsD}$.
\\
\emph{Induction step}: 
If the length of the call chain is $n+1$ (with $n\geq 0$),
then we may assume $n \in {\callsD}$.
By definitions  {\ref{defn-wreachable}.\ref{wr-calledge}}
and {\ref{defn-wreachable}.\ref{wr-start}}
we have that ${n+1} \in {\callsD}$.
\ignore{OLD TEXT WITH OLD DEF OF WR:
  If the length of the call chain, $n$,
  is greater than zero,  then assume 
  $i \in {\callsD}$ for all $call_i$ in the chain $i < n$.
  %It there is an $i<n$ such that $call_i$ is flooding the lemma holds.
  %Otherwise, 
  \Blue{We must then prove $n\in \callsD$.  There are two cases to
   consider.  First, if there is an $i<n$ such that $call_i$ is
   flooding then by the inductive hypothesis $i \in {\callsD}$ and there
   is a call chain from $call_i$ to $call_n$, satisfying the lemma.}
  In the other case, $call_n$ is the first call in the chain to be
  flooding.  By assumption the method with $start_{n-1}$ contains the
  call $call_n$.  $\forall i < n, {call_i}$, is not flooding thus
  $call_i$ is SR. \Blue{(I forget why not flooding implies SR. I guess
   we know it must be called and since it isn't flooding, it must
   finish. If it finishes its put is SR and that gets back to it being
   SR.  Do I need yet another lemma here?)}  and $\forall i < n,
 {call_i}$ is $SR$.

%(do I need to elaborate here? \Blue{YES, HOW ABOUT:
%  None of the  calls $call_i$ are flooding for $i<n$ and thus
%all start nodes  $start_i$  for $i<n$ are in SR}).

  By definition \ref{defn-wreachable} every node in the method
  containing $start_{n-1}$ is \emph{weakly-reachable} which includes
  the node $call_n$. Thus $call_n$ is \emph{weakly-reachable} and by
  definition \ref{defn-calls} is in ${\callsD}$.
}
\end{proof}

\subsection{Proof of Theorem \ref{thm-flooding}}
\begin{proof}
If there is some \emph{flooding} call, $call_n$, where $n \not\in {\callsD}-{\compsD}$ then either
\begin{description}
\item[a)] $n \not\in {\callsD}$, or
\item[b)] $n \in {\compsD}$.
\end{description}
By Lemma \ref{lemma-call-chain}, there must be a call chain starting on the 
cycle that leads to $call_n$ and by Lemma \ref{lemma-wr} $n \in {\callsD}$. 
%(Note: Need to correct of this being the first in a chain of flooded calls.)  
Therefore if the theorem does not hold, it must be because $n \in {\compsD}$ 
(and shouldn't be).

If $n \in {\compsD}$ then from definition \ref{defn-comps} and lemma
\ref{lemma-sr}, we know that either the method with label $n$ has
finished ($put_n$ or $get_n$ are SR), or the method with label $n$ is
partially executed directly in the cycle and none of the flow-paths
leading from the last node of the method in the cycle will suspend
before reaching a put.  Therefore $call_n$ cannot be flooding.
\end{proof}



\bibliographystyle{abbrv}
%\bibliographystyle{alpha} 
\bibliography{creol,ref}
%\bibliography{extracted}

\newpage
\section*{Appendix A}

\begin{figure}[ht]
%Example
\begin{abs}
Void cycle()      {1:m(); get$_1$; 2:cycle(); put}
Void m()$\ \ \ \ ${3:n(myfuture); put}
Void n(Future f)  {4:p; get$_4$; get$_f$; put}
\end{abs}
\caption{\label{example-quasi}
A  quasi example to illustrate/test page 9.
Here calls are already labeled and gets are indexed with
a label or future variable. \emph{myfuture} denotes the
future of the current call. The cycle should flood 3 calls.}
\end{figure}

\begin{figure}[h]
%Example
\begin{abs}
Void cycle()      {f:=1:m(); 2:n(f); get$_1$; Put}
Void m()$\ \ \ \ ${...}
Void n(Future f)  {get$_f$; 3: cycle(); put}
\end{abs}
\caption{\label{example-quasi2}
  Quasi example 2.
(As before calls are  labeled and gets are indexed with
a label or future variable.) 
%
Here there should be no real flooding since method
 \emph{cycle} will reach  \emph{put} since call 1 must be completed
due to the   \emph{get} in  \emph{n}.
CALLS: 1,2,3.  
COMPS: 1,2 and also 3 if we put back the (off-cycle)  flow edge into 
 $get_1$.
Without putting it back we do not reach \emph{put} of \emph{cycle}.}
\end{figure}

\begin{figure}[h]
%Example
\begin{abs}
Void cycle(f)      {1:m(f); get$_1$; 2:cycle(f); Put}
Void m(Future f)  {if (bExpr) {get$_f$; 3: n();}  put}
\end{abs}
\caption{\label{example-quasi3}
  Quasi example 3.
%Here 
Method
$n$ could be flooded if bExpr is true.%, but currently we don't mark the call to $n$ as reachable.
}
\end{figure}
\vfill
\end{document}
============================= end of file ====================================


output sets written after each statement, label before colon
  Sp  3 
1:Pd  1,3
2:Xp  1,2,3
           put-Sp
  Xp  1,2,3
  Get 1,~1,2,3
3:Cs  1,~1,2,~2,3
4:Sp  1,~1,2,~2,3
          put-Xp
