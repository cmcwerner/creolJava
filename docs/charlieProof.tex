\begin{lemma}
\label{lemma-sr}
If node $N_i$ is \emph{strongly-reachable} (SR) in graph $G$ with respect to cycle $C$ then $N_i$ is in every $T_{E,C,j}$ as defined in 
definition \ref{def-tree}\,.
\end{lemma}

\begin{proof}
We will use structural induction on the set of SR nodes to prove that Lemma \ref{lemma-sr} 
holds for all SR nodes as defined by definition \ref{defn-sreachable}\,.
\\ \emph%\subsubsection
{Base case:}
Let the set $S$ of SR nodes contain just nodes that are in $C$ (defn: \ref{defn-sreachable}.\ref{sr-incycle}).
Clearly all nodes in $S$ will be in $T_{E,C,j}$ by definition of $T_{E,C,m}$ (definition \ref{def-tree}\,).\footnote{What is $m$ in  $T_{E,C,m}$}
%
\\ \emph%subsubsection
{Induction step:}
Assume Lemma \ref{lemma-sr}  holds for some set $S$ of SR nodes. We will show that applying each of the subparts of defn: \ref{defn-sreachable}
defines a node to be SR for which the lemma holds.

\ref{defn-sreachable}.2: $N_j$ is not a blocking node. $N_i$ is in every $T_{E,C,k}$ by the inductive assumption. 
The only way to continue execution of the method containing $N_i$ is to execute $N_j$. Therefore 
$N_j$ must also be in $T_{E,C,k}$ (definition \ref{def-tree}.c).

\ref{defn-sreachable}.3: The argument is the same as for the previous case with the addition of the assumption that the \emph{put}
indicating the completion of any possible future needed by this \emph{get}
must also be SR (and in the tree). Since the only way to continue the execution of the method containing $N_i$
is $N_j$ and $N_j$ is unblocked due to the completion of its future, $N_j$ must be in $T_{E,C,i}$ (definition \ref{def-tree}.c).

\ref{defn-sreachable}.4: From the assumption of $n \in {\compsD}$, the method corresponding to
${start}_n$ has finished and it cannot finish without starting, thus
the node ${start}_n$ must have executed before the end of the tree. 
Also by assumption, the corresponding node ${call}_n$ is SR and in $T_{E,C,m}$, thus the node ${start}_n$ must also
be in $T_{E,C,m}$, satisfying the lemma.

\ref{defn-sreachable}.5: The nodes ${start}_n$ and $N_k$ (a ${get}_n$ node) are SR and in $T_{E,C,m}$ (by assumption). But ${get}_n$ can only execute
if ${put}_n$ has executed, thus the ${put}_n$ node $N_j$ must also be in $T_{E,C,m}$.

\ref{defn-sreachable}.6: The only way to get to node $N_k$ is through $N_j$ and by assumption $N_k$ is SR and in every $T_{E,C,m}$.
Thus the only way for $N_j$ to not be in $T_{E,C,m}$ would be for $N_k$ to be the first node in its method in $T_{E,C,m}$.
But this is impossible.
$N_k$ cannot be a \emph{start} node as \emph{start} nodes are not flow-reachable from any node. 
If $N_k$ is the first node in its method to execute in $T_{E,C,m}$ (i.e. although we must go through $N_j$ to get to $N_k$ maybe
$N_j$ is not in $T_{E,C,m}$), $N_k$ must be in the cycle which implies some other node $N_i$ reaches $N_k$ (to close the cycle) 
which contradicts the assumption that $N_k$ is not flow-reachable from any node other than $N_j$.
\end{proof}

\begin{lemma}
\label{lemma-call-chain}
If node $call_n$ is \emph{flooding} with respect to cycle $C$ in graph $G$ then there must be a call chain from a \emph{call}
in $C$ to $call_n$.\footnote{
Moreover, the start node of the  method called by e $call_n$
 is weakly reachable.}
\end{lemma}

\begin{proof}%Proof: 
This %The first part of this 
lemma follows directly from definition  \ref{flooding-cycle}\,.
\footnote{The second part follows by simple inductin on the length of the call chain
using  definition \ref{defn-wreachable}.\ref{wr-calledge}\,.}
\end{proof}

\begin{lemma}
\label{lemma-no-flood-chain}
If $\exists$ a call chain ${call}_0, {call}_1, ... {call}_n$ such that ${call}_0 \in C$ and all ${call}_i, i <= n$ are not flooding,
then all ${start}_i, i <= n$ are SR.
\end{lemma}

\begin{proof}%Proof: 
We will use structural induction on the length of the call chain.\\
\emph{Base case:} If the length of the call chain is zero, then $call_0$ is SR by definition \ref{defn-sreachable}.\ref{sr-incycle}\,.
Because $call_0$ is not flooding, $n$ must be in $\compsD$. Therefore
${start}_0$ is SR by definition \ref{defn-sreachable}.\ref{sr-start}\,.
\\
\emph{
Induction step:} If the length of the call chain, $n$, is greater than zero, then assume all ${start}_i, i < n$ are SR.
However, the only way for ${start}_i$ to be SR is for ${call}_i$ to be SR (definition \ref{defn-sreachable}.\ref{sr-start}), therefore
all ${call}_i, i < n$ are SR. Because ${start}_{n-1}$ is SR then by definition \ref{defn-wreachable}.\ref{wr-start} ${call}_n$ is weakly
reachable and thus $n \in {\callsD}$, and from the assumption of the lemma, ${call}_n$ is not flooding,
it must be the case that $n \in {\compsD}$. 
Therefore, by definition \ref{defn-sreachable}.\ref{sr-start}, 
${start}_n$ is SR.
\end{proof}

\begin{lemma}
\label{lemma-wr}
If node $call_n$ is \emph{flooding} with respect to cycle $C$ in graph $G$ then $n \in {\callsD}$\ignore{THE FOLLOWING PART CAN NOW BE REMOVED, RIGHT? or
$\exists call_i$ that is flooding, there is a call chain from $call_i$ to $call_n$, and $i \in {\callsD}$}.
\end{lemma}

\begin{proof}%Proof: 
We will use  induction on the length of the call chain from the call $call_0 \in C$ to $call_n$, where the
call chain is $call_0, call_1, ... call_n$.
\\
\emph{Base case}: 
If the length of the call chain is zero, then $call_0$ is the flooding
call which by definition \ref{defn-sreachable}.\ref{sr-incycle} is SR
and thus by definitions \ref{defn-wreachable} and \ref{defn-calls}, $0
\in {\callsD}$.
\\
\emph{Induction step}: 
If the length of the call chain is $n+1$ (with $n\geq 0$),
then we may assume $n \in {\callsD}$.
By definitions  {\ref{defn-wreachable}.\ref{wr-calledge}}
and {\ref{defn-wreachable}.\ref{wr-start}}
we have that ${n+1} \in {\callsD}$.
\ignore{OLD TEXT WITH OLD DEF OF WR:
  If the length of the call chain, $n$,
  is greater than zero,  then assume 
  $i \in {\callsD}$ for all $call_i$ in the chain $i < n$.
  %It there is an $i<n$ such that $call_i$ is flooding the lemma holds.
  %Otherwise, 
  \Blue{We must then prove $n\in \callsD$.  There are two cases to
   consider.  First, if there is an $i<n$ such that $call_i$ is
   flooding then by the inductive hypothesis $i \in {\callsD}$ and there
   is a call chain from $call_i$ to $call_n$, satisfying the lemma.}
  In the other case, $call_n$ is the first call in the chain to be
  flooding.  By assumption the method with $start_{n-1}$ contains the
  call $call_n$.  $\forall i < n, {call_i}$, is not flooding thus
  $call_i$ is SR. \Blue{(I forget why not flooding implies SR. I guess
   we know it must be called and since it isn't flooding, it must
   finish. If it finishes its put is SR and that gets back to it being
   SR.  Do I need yet another lemma here?)}  and $\forall i < n,
 {call_i}$ is $SR$.

%(do I need to elaborate here? \Blue{YES, HOW ABOUT:
%  None of the  calls $call_i$ are flooding for $i<n$ and thus
%all start nodes  $start_i$  for $i<n$ are in SR}).

  By definition \ref{defn-wreachable} every node in the method
  containing $start_{n-1}$ is \emph{weakly-reachable} which includes
  the node $call_n$. Thus $call_n$ is \emph{weakly-reachable} and by
  definition \ref{defn-calls} is in ${\callsD}$.
}
\end{proof}

\subsection{Proof of Theorem \ref{thm-flooding}}
\begin{proof}
If there is some \emph{flooding} call, $call_n$, where $n \not\in {\callsD}-{\compsD}$ then either
\begin{description}
\item[a)] $n \not\in {\callsD}$, or
\item[b)] $n \in {\compsD}$.
\end{description}
By Lemma \ref{lemma-call-chain}, there must be a call chain starting on the 
cycle that leads to $call_n$ and by Lemma \ref{lemma-wr} $n \in {\callsD}$. 
%(Note: Need to correct of this being the first in a chain of flooded calls.)  
Therefore if the theorem does not hold, it must be because $n \in {\compsD}$ 
(and shouldn't be).

If $n \in {\compsD}$ then from definition \ref{defn-comps} and lemma
\ref{lemma-sr}, we know that either the method with label $n$ has
finished ($put_n$ or $get_n$ are SR), or the method with label $n$ is
partially executed directly in the cycle and none of the flow-paths
leading from the last node of the method in the cycle will suspend
before reaching a put.  Therefore $call_n$ cannot be flooding.
\end{proof}
